\documentclass{beamer}
\usepackage{amsmath}
\usepackage[english]{babel} %set language; note: after changing this, you need to delete all auxiliary files to recompile
\usepackage[utf8]{inputenc} %define file encoding; latin1 is the other often used option
\usepackage{csquotes} % provides context sensitive quotation facilities
\usepackage{graphicx} %allows for inserting figures
\usepackage{booktabs} % for table formatting without vertical lines
\usepackage{textcomp} % allow for example using the Euro sign with \texteuro
\usepackage{stackengine}
\usepackage{wasysym}
\usepackage{tikzsymbols}
\usepackage{textcomp}
\newcommand{\bubblethis}[2]{
        \tikz[remember picture,baseline]{\node[anchor=base,inner sep=0,outer sep=0]%
        (#1) {\underline{#1}};\node[overlay,cloud callout,callout relative pointer={(0.2cm,-0.7cm)},%
        aspect=2.5,fill=yellow!90] at ($(#1.north)+(-0.5cm,1.6cm)$) {#2};}%
    }%
\tikzset{face/.style={shape=circle,minimum size=4ex,shading=radial,outer sep=0pt,
        inner color=white!50!yellow,outer color= yellow!70!orange}}
%% Some commands to make the code easier
\newcommand{\emoticon}[1][]{%
  \node[face,#1] (emoticon) {};
  %% The eyes are fixed.
  \draw[fill=white] (-1ex,0ex) ..controls (-0.5ex,0.2ex)and(0.5ex,0.2ex)..
        (1ex,0.0ex) ..controls ( 1.5ex,1.5ex)and( 0.2ex,1.7ex)..
        (0ex,0.4ex) ..controls (-0.2ex,1.7ex)and(-1.5ex,1.5ex)..
        (-1ex,0ex)--cycle;}
\newcommand{\pupils}{
  %% standard pupils
  \fill[shift={(0.5ex,0.5ex)},rotate=80] 
       (0,0) ellipse (0.3ex and 0.15ex);
  \fill[shift={(-0.5ex,0.5ex)},rotate=100] 
       (0,0) ellipse (0.3ex and 0.15ex);}

\newcommand{\emoticonname}[1]{
  \node[below=1ex of emoticon,font=\footnotesize,
        minimum width=4cm]{#1};}
\usepackage{scalerel}
\usetikzlibrary{positioning}
\usepackage{xcolor,amssymb}
\newcommand\dangersignb[1][2ex]{%
  \scaleto{\stackengine{0.3pt}{\scalebox{1.1}[.9]{%
  \color{red}$\blacktriangle$}}{\tiny\bfseries !}{O}{c}{F}{F}{L}}{#1}%
}
\newcommand\dangersignw[1][2ex]{%
  \scaleto{\stackengine{0.3pt}{\scalebox{1.1}[.9]{%
  \color{red}$\blacktriangle$}}{\color{white}\tiny\bfseries !}{O}{c}{F}{F}{L}}{#1}%
}
\usepackage{fontawesome} % Social Icons
\usepackage{epstopdf} % allow embedding eps-figures
\usepackage{tikz} % allows drawing figures
\usepackage{amsmath,amssymb,amsthm} %advanced math facilities
\usepackage{lmodern} %uses font that support italic and bold at the same time
\usepackage{hyperref}
\usepackage{tikz}
\hypersetup{
    colorlinks=true,
    linkcolor=blue,
    filecolor=magenta,      
    urlcolor=blue,
}
\usepackage{tcolorbox}
%add citation management using BibLaTeX
\usepackage[citestyle=authoryear-comp, %define style for citations
    bibstyle=authoryear-comp, %define style for bibliography
    maxbibnames=10, %maximum number of authors displayed in bibliography
    minbibnames=1, %minimum number of authors displayed in bibliography
    maxcitenames=3, %maximum number of authors displayed in citations before using et al.
    minnames=1, %maximum number of authors displayed in citations before using et al.
    datezeros=false, % do not print dates with leading zeros
    date=long, %use long formats for dates
    isbn=false,% show no ISBNs in bibliography (applies only if not a mandatory field)
    url=false,% show no urls in bibliography (applies only if not a mandatory field)
    doi=false, % show no dois in bibliography (applies only if not a mandatory field)
    eprint=false, %show no eprint-field in bibliography (applies only if not a mandatory field)
    backend=biber %use biber as the backend; backend=bibtex is less powerful, but easier to install
    ]{biblatex}
\addbibresource{../mybibfile.bib} %define bib-file located one folder higher


\usefonttheme[onlymath]{serif} %set math font to serif ones

\definecolor{beamerblue}{rgb}{0.2,0.2,0.7} %define beamerblue color for later use

%%% defines highlight command to set text blue
\newcommand{\highlight}[1]{{\color{blue}{#1}}}


%%%%%%% commands defining backup slides so that frame numbering is correct

\newcommand{\backupbegin}{
   \newcounter{framenumberappendix}
   \setcounter{framenumberappendix}{\value{framenumber}}
}
\newcommand{\backupend}{
   \addtocounter{framenumberappendix}{-\value{framenumber}}
   \addtocounter{framenumber}{\value{framenumberappendix}}
}

%%%% end of defining backup slides

%Specify figure caption, see also http://tex.stackexchange.com/questions/155738/caption-package-not-working-with-beamer
\setbeamertemplate{caption}{\insertcaption} %redefines caption to remove label "Figure".
%\setbeamerfont{caption}{size=\scriptsize,shape=\itshape,series=\bfseries} %sets figure  caption bold and italic and makes it smaller


\usetheme{Boadilla}

%set options of hyperref package
\hypersetup{
    bookmarksnumbered=true, %put section numbers in bookmarks
    naturalnames=true, %use LATEX-computed names for links
    citebordercolor={1 1 1}, %color of border around cites, here: white, i.e. invisible
    linkbordercolor={1 1 1}, %color of border around links, here: white, i.e. invisible
    colorlinks=true, %color links
    anchorcolor=black, %set color of anchors
    linkcolor=beamerblue, %set link color to beamer blue
    citecolor=blue, %set cite color to beamer blue
    pdfpagemode=UseThumbs, %set default mode of PDF display
    breaklinks=true, %break long links
    pdfstartpage=1 %start at first page
    }


% --------------------
% Overall information
% --------------------
\title[Economía I]{Economía I \vspace{4mm}
\\ Magistral 23: Mercado de dinero}
\date{}
\author[Ertola Navajas y Fariña]{Ertola Navajas y Fariña}
\vspace{0.4cm}
\institute[]{Universidad de San Andrés} 


\begin{document}

\begin{frame}
\titlepage
\centering
Magistral 23

\includegraphics[scale=0.2]{Slides Principios de Economia/Figures/logoUDESA.jpg} 
\end{frame}



\begin{frame}{Las funciones del dinero}
    \begin{itemize}
       \item \textbf{Medio de cambio}: el dinero es utilizado como un mecanismo para realizar transacciones. 
  \vspace{1mm}
    \item \textbf{Unidad de cuenta}: el dinero es también una medida que sirve para definir precios así como para registrar activos y deudas.
  \vspace{1mm}  
    \item \textbf{Depósito de valor}: el dinero permite transferir poder adquisitivo del presente al futuro. 
   \end{itemize}
\end{frame}

\begin{frame}
\frametitle{Tipos de dinero}
\begin{itemize}
    \item Durante la mayor parte de la historia de la humanidad se utilizó dinero mercancía (commodity money)
        \begin{itemize}
        \item Mercancías con valor intrínseco que se usaban para comerciar \\
        - Oro, plata, cigarrillos, etc.
        \end{itemize} \vspace{1mm}
    \item Hoy en día, casi todo el dinero es fiduciario (fiat money)               \begin{itemize}
        \item Sin valor intrínseco, pero que se utiliza porque el gobierno lo hace de curso legal (legal tender) \\
        - Pesos, euros, dólares, bonos provinciales (?!), etc.
        \item Ahora es casi natural, pero tomó mucho tiempo \\
        - Problemas de confianza, falsificación, funcionamiento, etc.
        \end{itemize}
\end{itemize}
\end{frame}

\begin{frame}{Commodity money }
            \begin{figure} [H]   
  \centering
  \includegraphics[width=.8\textwidth]{Slides Principios de Economia/Figures/C32.1.jpg}
      \caption{Monedas carcomidas}
  \label{fig:C32.1}
\end{figure}
\end{frame}

\begin{frame}{Dinero Papel}
    \begin{itemize}
        \item Comienza a usarse en China en el Siglo VII
        \item Antes de los bancos centrales los emitían los bancos 
        
\begin{figure} [H]   
  \centering
  \includegraphics[width=.35\textwidth]{Slides Principios de Economia/Figures/C32.2.jpg}
      \caption{Dollar de Bank of Rahway, New Jersey, 1850. Via Wikimedia Commons}
  \label{fig:C32.2}
\end{figure}

\begin{figure} [H]   
  \centering
  \includegraphics[width=.35\textwidth]{Slides Principios de Economia/Figures/C32.3.jpeg}
      \caption{Detroit City Bank \$3 Note}
  \label{fig:C32.3}
\end{figure}
\end{itemize}
\end{frame}
        
\begin{frame}{El dinero}
        \begin{figure} [H]   
  \centering
  \includegraphics[width=.35\textwidth]{Slides Principios de Economia/Figures/C32.4.jpg}
      \caption{Billetes emitidos por el Banco Provincia de Buenos Aires}
  \label{fig:C32.4}
\end{figure}

\begin{figure} [H]   
\centering\includegraphics[width=.35\textwidth]{Slides Principios de Economia/Figures/C32.5.jpg}
\caption{Billetes emitidos por el Banco Provincia de Buenos Aires}
\end{figure}
       \begin{itemize}
           \item  Luego lo monopolizaron los bancos centrales
    \item Y hoy volvemos a la multiplicidad de monedas 
       \end{itemize}     
        \end{frame}


\begin{frame}{La demanda de dinero}
    \begin{itemize}
        \item La demanda de dinero es lo que la gente demanda de dinero, es cuánto dinero desea tener la sociedad
        \item Se explica por dos motivos principalmente: por motivo transacción y por motivo especulación
        \begin{itemize}
        \item Por transacciones $\Rightarrow P*Y \Rightarrow$ a mayores precios o mayor nivel de actividad aumenta la demanda de dinero por transacciones
        \item Costo de oportunidad $\Rightarrow i \Rightarrow $ la demanda de dinero depende negativamente de la tasa de interés que es el costo de oportunidad del dinero
    \end{itemize}
    \item La llegada de las tarjetas de crédito, el dinero bancario y la agilidad de transferir dinero de activos líquidos a dinero, entre otros, son elementos que han ido cambiando drásticamente la demanda de dinero en el tiempo
    \end{itemize}
\end{frame}

\begin{frame}{La oferta de dinero}
    \begin{itemize}
       \item Oferta primaria. Determinada por el Banco Central que incluye en la base monetaria.
       \item Oferta secundaria. Dinero creado por los bancos comerciales al otorgar créditos a partir de depósitos.
       \end{itemize}
\end{frame}

\begin{frame}
\frametitle{¿Qué es el Banco Central?}
\begin{itemize}
    \item Un banco particular
        \begin{itemize}
            \item Es generalmente propiedad del gobierno
            \item Actúa como banquero de los bancos comerciales \\
            - Que tienen `reservas' en el Banco Central
            \item Es el único que puede crear moneda de curso legal
        \end{itemize}
    \item Dinero en el sentido amplio
        \begin{itemize}
            \item Base monetaria (base money) o M0 \\
            - Billetes y monedas, más cuentas depositadas en el Banco Central
            \item Depósitos (bank money) \\
            - Dinero creado por los bancos comerciales al extender crédito
        \end{itemize}
\end{itemize}
\end{frame}

\begin{frame}
\frametitle{¿Qué es el Banco Central?}
\begin{itemize}
    \item Un banco particular
        \begin{itemize}
            \item Es generalmente propiedad del gobierno
            \item Actúa como banquero de los bancos comerciales \\
            - Que tienen `reservas' en el Banco Central
            \item Es el único que puede crear moneda de curso legal
        \end{itemize}
    \item Dinero en el sentido amplio
        \begin{itemize}
            \item Base monetaria (base money) o M0 \\
            - Billetes y monedas, más cuentas depositadas en el Banco Central
            \item Depósitos (bank money) \\
            - Dinero creado por los bancos comerciales al extender crédito
        \end{itemize}
\end{itemize}
\end{frame}

\begin{frame}{El balance del Banco Central}
\begin{figure} [H] 
\centering
\includegraphics[width=6.5cm]{Slides Principios de Economia/Figures/P48.png}\
\end{figure}

\end{frame}

\begin{frame}{¿Qué más hace el Banco Central?}
    \begin{itemize}
    \item Crea y regula la cantidad de dinero en la economía
    \vspace{1mm}
    \item Regula la actividad bancaria
    \vspace{1mm}
    \item Es prestamista de última instancia
    \vspace{1mm}
    \item Maneja la política cambiaria 
\end{itemize}
\end{frame}

\begin{frame}
\frametitle{¿Qué son los bancos?}
\begin{itemize}
    \item Son intermediarios financieros
        \begin{itemize}
        \item Instituciones que reciben fondos de personas y empresas, y los utilizan para comprar bonos o acciones, o para hacer préstamos a otros agentes \\
        - Los bancos piden prestado a los hogares (depósitos), otros bancos, y el banco central \\
        - El interés que pagan por los depósitos (tasa de interés pasiva) es menor que el que cobran en préstamos (tasa de interés activa, lending rate), y así obtienen beneficios
        \end{itemize}
    \item Un tipo particular de intermediario financiero
    ¡sus pasivos son dinero!
\end{itemize}
\end{frame}

\begin{frame}{Riesgos que tienen que manejar los bancos}
\begin{itemize}
    \item \textbf{Riesgo de madurez}: como el banco invierte en activos de largo plazo, si la tasa de interés sube, en general el valor de los activos de largo plazo va a caer más que los de corto plazo \vspace{1mm}
    \item \textbf{Riesgo de liquidez}: Es el riesgo de que el activo no se pueda transformar en efectivo (liquidar) sin generar una pérdida financiera \vspace{1mm}
    \item \textbf{Riesgo de default}: Es el riesgo de que los créditos del banco no sean repagados
\end{itemize}
    \end{frame}

\begin{frame}{Aplancamiento (leverage)}
    \begin{itemize}
    \item Un banco es solvente cuando el valor de sus activos supera al de los pasivos
     \item ¿Cómo evaluamos el riesgo de una caida en el valor de los activos?
    \end{itemize}
     \begin{center}
       $ Leverage = \frac{Activo}{\text{Patrimonio Neto}} $
    \end{center} 
    \begin{itemize}
    \item Un coeficiente muy alto, es decir, mucho apalancamiento, implica que gran parte de los activos son financiados con deuda y poco con patrimonio neto. Es decir que los activos se financian con dinero de terceros, no con capital propio.
    \end{itemize}
\end{frame}

\begin{frame}{Ejemplo I}
    \begin{table}[H]
    \centering
%    \vspace{.3cm}
    %\scalebox{.8}{
    \begin{tabular}{|c|c|c|c|}
    \hline
\textbf{Banco}    & \textbf{Activo} & \textbf{Pasivo} & \textbf{Patrimonio Neto}\\
         \hline \hline
         Banco 1 &  100 &  80 & 20\\[1mm]
        \hline
       Banco 2 & 100  &  95& 5\\[1mm]
        \hline
    \end{tabular}
    %}
    \caption{Escenario Inicial}
    \label{inicial}
\end{table}
\end{frame}

\begin{frame}{Ejemplo II}
   \begin{table}[H]
    \centering

%    \vspace{.3cm}
    %\scalebox{.8}{
    \begin{tabular}{|c|c|c|c|}
    \hline
\textbf{Banco}    & \textbf{Activo} & \textbf{Pasivo} & \textbf{Patrimonio Neto}\\
         \hline \hline
         Banco 1 &  90 &  80 & 10\\[1mm]
        \hline
       Banco 2 & 90  &  95& -5\\[1mm]
        \hline
    \end{tabular}
    %}
    \caption{Caída del 10\% del valor de los activos}
    \label{caida10pp}
\end{table} 
\end{frame} 

\begin{frame}{¿Por qué uno quiere leverage?}
\begin{table}[H]
    \centering
%    \vspace{.3cm}
    %\scalebox{.8}{
    \begin{tabular}{|c|c|c|c|}
    \hline
\textbf{Banco}    & \textbf{Activo} & \textbf{Pasivo} & \textbf{Patrimonio Neto}\\
         \hline \hline
         Banco 1 &  110 &  80 & 30\\[1mm]
        \hline
       Banco 2 & 110  &  95& 15\\[1mm]
        \hline
    \end{tabular}
    %}
    \caption{Aumento del 10\% del valor de los activos}
    \label{caida10pp}
\end{table}
\end{frame}

\begin{frame}{Mercado monetario: oferta de dinero \\ La hoja de balance del Banco Central }
\centering\includegraphics[width=6.5cm]{Slides Principios de Economia/Figures/P48.png}\
\end{frame}


\begin{frame}{El multiplicador monetario}
    \begin{itemize}
        \item Ejemplo de creación de dinero bancario:  circulante = 100  y  encajes = 10\%
    \end{itemize}

    \vspace{2mm}
    
    \centering\includegraphics[width=10cm]{Slides Principios de Economia/Figures/P50.png}\
    
    \vspace{2mm}
    
    \begin{tcolorbox}[width=4in, interior hidden, boxsep=0pt,
                  left=0pt, halign=center, valign=center, right=0pt,
                  bottom=3pt, top=3pt, ]%%
                 \footnotesize{M1 = circulante + depósitos = 100 + 90 + 81 + 72.9 + ..... = 1000}
    \end{tcolorbox}
    \vspace{2mm}
\end{frame}

\begin{frame}{El multiplicador monetario}
    \begin{itemize}
        \item Ejemplo de creación de dinero bancario:  circulante = 100  y  encajes = 20\%
    \end{itemize}
        \vspace{2mm}
    \centering\includegraphics[width=10cm]{Slides Principios de Economia/Figures/P51.png}\
    \vspace{2mm}
    \begin{tcolorbox}[width=4in, interior hidden, boxsep=0pt,
                  left=0pt, halign=center, valign=center, right=0pt,
                  bottom=3pt, top=3pt, ]%%
                 \footnotesize{M1 = circulante + depósitos = 100 + 80 + 64 + 51.2 + ..... = 500}
    \end{tcolorbox}
        \vspace{2mm}
    \begin{itemize}
        \item Los encajes se usan para regular la cantidad de dinero
        \item ¡Subimos los encajes y cae la cantidad de dinero!  \Large\Cooley[][yellow!60!white]
        \end{itemize}
\end{frame}


\begin{frame}{Los agregados monetarios M}
     \begin{itemize}
        \item M0 = Base monetaria = Circulante + Encajes bancarios
        \item M1 = Circulante + Depósitos en cuenta corriente
        \item M2 = M1 + Caja de Ahorro
        \item M3 = M2 + Plazo Fijo
        \item M? = M3 + saldos de tarjetas?, programa de millajes? 
        \item etc.
    \end{itemize}
\end{frame}

\end{document}