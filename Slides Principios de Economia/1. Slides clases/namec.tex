\documentclass[14pt]{beamer}
\usepackage[utf8]{inputenc}
\usetheme{Singapore}
\usepackage{amsmath}
\usepackage{amsfonts}
\usepackage{amssymb}
\usepackage{amsmath}
\usepackage{graphicx} 
\usepackage{caption}
\usepackage{subcaption}
\hypersetup{
    colorlinks=true,
    linkcolor=blue,
    filecolor=magenta,      
    urlcolor=blue,
}
\title{ECONOM\'{I}A I (E010)}
\subtitle{Tema 1 \\ Introducción}
\setbeamertemplate{navigation symbols}{}


\begin{document}

\begin{frame}
\frametitle{ECONOM\'{I}A I (E010) \\ \vspace{12mm} Magistral 17}
\centering
Tema 8 \\ Distorsiones al equilibrio de mercado \\ \vspace{4mm}
\includegraphics[scale=0.25]{logoUDESA.jpg} 
\end{frame}

\begin{frame}{32. Asimetrías de información}
    \begin{itemize}
        \item \textbf{Riesgo moral o acción oculta} alguien no puede ver las acciones del otro
            \begin{itemize}
            \item Ejemplos: seguros en general, incentivos a ahorrar, etc. 
            \end{itemize}
        \item \textbf{Selección adversa} Es cuando no conocemos una característica de la contraparte (atributos ocultos)
            \begin{itemize}
            \item Ejemplos: seguros en general, usados. 
            \end{itemize}
        %\item El resultado es que el mercado incluso puede desaparecer!
    \end{itemize}
    \end{frame}
    
\begin{frame}{33. El mercado de los limones de Akerlof}
\begin{itemize}
    \item Dos tipos de autos: buenos y fallados
    \item q son buenos y (1-q) son malos 
    \item Para el vendedor el auto bueno vale 1000 y el lemon 500 
    \item Para el comprador el auto bueno vale 1500 y el lemon 750 
    \item Vamos a ver cuanto estaría dispuesto a pagar un comprador y dado eso después vemos que le conviene hacer al vendedor
    \item El comprador esta dispuesta a pagar seria su valor esperado
    \item Si piensa que la probabilidad que un auto sea bueno sea  $\mu$ y que sea malo  $(1-\mu)$ el valor esperado para un auto típico en el mercado sería  $\mu 1500 + (1-\mu) 750= 750 + \mu 750$ 
    \item El vendedor venderá un auto si lo que cobra por él supera lo que el lo valúa
\end{itemize}
\end{frame}

\begin{frame}{34. El mercado de los limones II}
\begin{itemize}
    \item Si tiene un lemon lo venderá si 
    $500 \leq 750 + \mu 750$
    \item Que se da siempre. Lo cual quiere decir que si el vendedor tiene un lemon lo pone en el mercado siempre
    \item Si tiene un auto bueno, lo venderá si 
    $1000 \leq 750 + \mu 750 $ (1)
    \item Esto solo se daría si $\mu \geq \frac{1}{3}$ 
\end{itemize}
\end{frame}

\begin{frame}{35. El mercado de los limones III}
\begin{itemize}
\item  Si $ q \geq \frac{1}{3}$, (hay suficiente buenos) hay un equilibrio donde $\mu = q \geq \frac{1}{3}$ y $p= 750 + \mu 750$ y se venden los dos tipos de autos. 
\item Pero si $q \leq \frac{1}{3}$ entonces por la ecuación (1) sabemos que $\mu=0 $, lo cual quiere decir que se venden solo limones y el precio de venta es de $p= 750$. 
\item El mercado para autos buenos desapareció aun cuando dijimos al principio que estos tenían más valor para los consumidores que para los vendedores. %¡La mano invisible de Adam Smith no pudo operar por la asimetría de información!

\end{itemize} 
\end{frame}

\begin{frame}{36. Riesgo moral y el colapso del mercado de seguros}
 \begin{itemize}
     \item Un individuo que tiene que comprar un seguro de incendio para su casa. 
     \item Pueden pasar dos cosas. O que la casa no se incendie (llamemos a este evento B, por bueno) o que la casa se incendie (llamemos a este evento M por malo). 
    \item B sucede con probabilidad $p$ y el evento malo con probabilidad $(1-p)$. 
    \item Si se produce el evento malo, el individuo sufre una perdida de tamaño $L$
    \item La la probabilidad del evento bueno depende en parte de alguna acción del individuo, $p(e)$.
    \item Quien ofrece el seguro no puede ver esta acción. 
\end{itemize}
\end{frame}

\begin{frame}{37. Riesgo moral II}
    \begin{itemize}
    \item Veamos que pasa si la compañía ofrece una cobertura de valor $C$ a un precio $\pi$. 
    \item Si pasa el escenario bueno la utilidad para el individuo es 
    $Y_B = y - \pi C$
    \item Si se produce el evento malo su utilidad es 
    $Y_M = y - L - \pi C + C$
    \item ¿Qué $\pi$ podrían cobrar la compañía de seguros? Vamos a asumir el caso más favorable al asegurador, que es que la compañías de seguro salen hechas. 
    \item La ecuación de ganancias de las aseguradoras es  
    $\pi C - (1-p) C$
    \item Si esta ganancia la hacemos $0$ el (menor) precio que puede cobrar la compañía es $\pi=(1-p)$
    \end{itemize}
\end{frame}

\begin{frame}{38. Riesgo moral III}
    \begin{itemize}
    \item Si el individuo compra una cobertura de $C=L$ en el evento bueno su utilidad es $y-\pi L$ y en el evento malo es $y-L-\pi L+L=y-\pi L $
    \item Al comprar cobertura total, le es indiferente si se produce el evento bueno o el malo. 
    \item Pero entonces $e=0$ y la probabilidad del siniestro va a ser más alta. 
    \item Consideremos el caso en el cual si el individuo no hace nada el siniestro ocurre con probabilidad 1. En ese caso $p=0$ y $\pi=1$
    \item La utilidad para el individuo de comprar seguro es $y-L$, que resulta peor que no comprar seguro. 
    \item ¡Es decir que el mercado asegurador desaparece! 
  \end{itemize}   
\end{frame}


\begin{frame}{39. Discusión}
    \begin{itemize}
        \item ¿Por qué pierden valor los autos al salir de la concesionaria?
        \item Políticas de deducibles
        \item Obama care
    \end{itemize}
\end{frame}



\end{document}
    
    
    
    
    