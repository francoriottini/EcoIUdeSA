\documentclass{beamer}
\usepackage{amsmath}
\usepackage[english]{babel} %set language; note: after changing this, you need to delete all auxiliary files to recompile
\usepackage[utf8]{inputenc} %define file encoding; latin1 is the other often used option
\usepackage{csquotes} % provides context sensitive quotation facilities
\usepackage{graphicx} %allows for inserting figures
\usepackage{booktabs} % for table formatting without vertical lines
\usepackage{textcomp} % allow for example using the Euro sign with \texteuro
\usepackage{stackengine}
\usepackage{wasysym}
\usepackage{tikzsymbols}
\usepackage{textcomp}
\newcommand{\bubblethis}[2]{
        \tikz[remember picture,baseline]{\node[anchor=base,inner sep=0,outer sep=0]%
        (#1) {\underline{#1}};\node[overlay,cloud callout,callout relative pointer={(0.2cm,-0.7cm)},%
        aspect=2.5,fill=yellow!90] at ($(#1.north)+(-0.5cm,1.6cm)$) {#2};}%
    }%
\tikzset{face/.style={shape=circle,minimum size=4ex,shading=radial,outer sep=0pt,
        inner color=white!50!yellow,outer color= yellow!70!orange}}
%% Some commands to make the code easier
\newcommand{\emoticon}[1][]{%
  \node[face,#1] (emoticon) {};
  %% The eyes are fixed.
  \draw[fill=white] (-1ex,0ex) ..controls (-0.5ex,0.2ex)and(0.5ex,0.2ex)..
        (1ex,0.0ex) ..controls ( 1.5ex,1.5ex)and( 0.2ex,1.7ex)..
        (0ex,0.4ex) ..controls (-0.2ex,1.7ex)and(-1.5ex,1.5ex)..
        (-1ex,0ex)--cycle;}
\newcommand{\pupils}{
  %% standard pupils
  \fill[shift={(0.5ex,0.5ex)},rotate=80] 
       (0,0) ellipse (0.3ex and 0.15ex);
  \fill[shift={(-0.5ex,0.5ex)},rotate=100] 
       (0,0) ellipse (0.3ex and 0.15ex);}

\newcommand{\emoticonname}[1]{
  \node[below=1ex of emoticon,font=\footnotesize,
        minimum width=4cm]{#1};}
\usepackage{scalerel}
\usetikzlibrary{positioning}
\usepackage{xcolor,amssymb}
\newcommand\dangersignb[1][2ex]{%
  \scaleto{\stackengine{0.3pt}{\scalebox{1.1}[.9]{%
  \color{red}$\blacktriangle$}}{\tiny\bfseries !}{O}{c}{F}{F}{L}}{#1}%
}
\newcommand\dangersignw[1][2ex]{%
  \scaleto{\stackengine{0.3pt}{\scalebox{1.1}[.9]{%
  \color{red}$\blacktriangle$}}{\color{white}\tiny\bfseries !}{O}{c}{F}{F}{L}}{#1}%
}
\usepackage{fontawesome} % Social Icons
\usepackage{epstopdf} % allow embedding eps-figures
\usepackage{tikz} % allows drawing figures
\usepackage{amsmath,amssymb,amsthm} %advanced math facilities
\usepackage{lmodern} %uses font that support italic and bold at the same time
\usepackage{hyperref}
\usepackage{tikz}
\usepackage{tcolorbox}

\usefonttheme[onlymath]{serif} %set math font to serif ones

\definecolor{beamerblue}{rgb}{0.2,0.2,0.7} %define beamerblue color for later use

%%% defines highlight command to set text blue
\newcommand{\highlight}[1]{{\color{blue}{#1}}}


%%%%%%% commands defining backup slides so that frame numbering is correct

\newcommand{\backupbegin}{
   \newcounter{framenumberappendix}
   \setcounter{framenumberappendix}{\value{framenumber}}
}
\newcommand{\backupend}{
   \addtocounter{framenumberappendix}{-\value{framenumber}}
   \addtocounter{framenumber}{\value{framenumberappendix}}
}

%%%% end of defining backup slides

%Specify figure caption, see also http://tex.stackexchange.com/questions/155738/caption-package-not-working-with-beamer
\setbeamertemplate{caption}{\insertcaption} %redefines caption to remove label "Figure".
%\setbeamerfont{caption}{size=\scriptsize,shape=\itshape,series=\bfseries} %sets figure  caption bold and italic and makes it smaller

\newtcolorbox{boxA}{
    fontupper = \bf,
    boxrule = 1.5pt,
    colframe = black % frame color
}

\usetheme{Boadilla}


% --------------------
% Overall information
% --------------------
\title[Economía I]{Economía I \vspace{4mm}
\\ Magistral 17: Distorsiones de mercado III}
\date{}
\author[Riottini]{Riottini Franco}
\vspace{0.4cm}
\institute[]{Universidad de San Andrés} 


\begin{document}

\begin{frame}
\titlepage
\centering
\includegraphics[scale=0.2]{../Figures/logoUDESA.jpg} 
\end{frame}

\begin{frame}{Distorsiones al equilibrio de mercado}
    \begin{itemize}
        \item Por las características de la realidad \vspace{1mm}
        \begin{itemize}
            \item Monopolios naturales (red eléctrica, agua, gas)   
             \vspace{1mm}
            \item Externalidades
             \vspace{1mm}
            \item Bienes públicos
            \vspace{1mm}
            \item Problemas de información
            \begin{itemize}
                \item Atributos ocultos (selección adversa)
                 \vspace{1mm}
                \item Acciones ocultas (moral hazard)
            \end{itemize}        
        \end{itemize}
    \end{itemize}
\end{frame}

\begin{frame}{Asimetrías de información}
    \begin{boxA}
        \centering
        Hay información asimétrica
        cuando una de las partes tiene información de importancia para una
        transacción que la otra parte desconoce.
    \end{boxA}
\end{frame}

\begin{frame}
    \textbf{Selección adversa} sucede cuando no conocemos una característica de la contraparte (atributos ocultos).
    \begin{itemize}
        \item Ejemplos: seguros en general, mercado de usados, empresas buscando contratar, etc.  
        \item No podemos saber que gente se enferma mas que otra, no podemos saber que auto está en peor estado que otro (o solo una aproximación), no podemos saber si un trabajador es eficiente o no, etc.
    \end{itemize}
    \textbf{Riesgo moral} alguien no puede ver las acciones del otro.
    \begin{itemize}
        \item Ejemplos: seguros en general, incentivos a ahorrar, incentivos a trabajar, etc.
        \item No podemos ver si la gente se cuida para no enfermarse, no podemos ver si el conductor maneja con cuidado, no podemos ver si la gente ahorra lo suficiente como para pagar sus deudas, no podemos ver si el empleado se esfuerza, etc.
    \end{itemize}
\end{frame}

\begin{frame}{Ejemplos}
    \begin{itemize}
    \item \href{https://www.youtube.com/watch?v=qlg0qakJhKU}{\textbf{Matilda}}
    \item \href{https://www.youtube.com/watch?v=akA8co61He4}{\textbf{Tomates verdes fritos}}
    \item \href{https://www.youtube.com/watch?v=X8BPfLhH6MA}{\textbf{Friends}}
    \item \href{https://videos.criticalcommons.org/media/encoded/16/jtierney86/43ba1b1ac3e94df3974f987cc912ae_Hxgbfl1.mp4}{\textbf{The Daily Show}}
    \item \href{http://videos.criticalcommons.org/transcoded/http/www.criticalcommons.org/Members/JJWooten/clips/always-sunny-paying-for-care/video_file/mp4-high/always-sunny-cost-of-care-mp4.mp4}{\textbf{Always sunny}}
    \item \href{https://www.youtube.com/watch?v=SrPu-xGrKrk}{\textbf{Buying a car}}
    \item \href{https://www.youtube.com/watch?v=ZZq0ShjEd-E}{\textbf{But he has a Bud Light}}
    \end{itemize}
\end{frame}
    
\begin{frame}{El mercado de los limones de Akerlof}
    \begin{itemize}
        \item Dos tipos de autos: buenos $(q)$ y malos, limones, $(1 - q)$
        \begin{itemize}
            \item Para el vendedor el auto bueno vale 1000 y el malo 500.
            \item Para el comprador el auto bueno vale 1500 y el malo 750.
            \item ¿A cuanto se venden los autos si sabemos cual es cual? 
        \end{itemize}
        \item Vamos a ver cuanto estaría dispuesto a pagar un comprador y dado eso después vemos que le conviene hacer al vendedor
        \begin{itemize}
            \item El comprador \textbf{está dispuesto a pagar su valor esperado}, pero no sabe cuanto es $q$ (o se puede pensar que en el mercado puede que $q \neq \mu$).
            \item Si piensa que la probabilidad que un auto sea bueno sea  $\mu$ y que sea malo  $(1-\mu)$ el valor esperado para un auto típico en el mercado sería  $\mu 1500 + (1-\mu) 750= 750 + \mu 750$ 
        \end{itemize}
    \end{itemize}
\end{frame}

\begin{frame}{El mercado de los limones de Akerlof}
    \begin{itemize}
        \item El vendedor venderá un auto si lo que cobra por él supera su propia valoración:
        \item Si tiene un auto malo lo venderá si 
        \begin{equation*}
            500 \leq 750 + \mu 750
        \end{equation*}
        Esto se da siempre: quiere decir que si el vendedor tiene un lemon lo pone en el mercado siempre
        \item Si tiene un auto bueno, lo venderá si 
        \begin{equation} \label{eq:1}
            1000 \leq 750 + \mu 750
        \end{equation}
        Esto solo se daría si $\mu \geq \frac{1}{3}$
    \end{itemize}
\end{frame}

\begin{frame}{El mercado de los limones de Akerlof}
\begin{itemize}
    \item  Si $ q \geq \frac{1}{3}$, (hay suficiente buenos) hay un equilibrio donde $\mu = q \geq \frac{1}{3}$ y $p= 750 + \mu 750$ y se venden los dos tipos de autos 
    \vspace{1mm}
    \item Pero si $q \leq \frac{1}{3}$ entonces por la ecuación \eqref{eq:1} sabemos que el auto no se vende
    \begin{itemize}
    \item El vendedor no tiene incentivos a tener autos nuevos, puesto que no los vendería en este caso
    \item Si $q= \mu = 0 $, quiere decir que se venden sólo limones y el precio de venta es de $p= 750$ 
    \end{itemize}
    \item El mercado para autos buenos desapareció aun cuando dijimos al principio que estos tenían más valor para los consumidores que para los vendedores \vspace{1mm}
    \item ¡La mano invisible de Adam Smith no pudo operar por la asimetría de información!
\end{itemize} 
\end{frame}

\begin{frame}{Riesgo moral y el colapso del mercado de seguros}
    \begin{itemize}
        \item Imaginemos una persona que tiene que comprar un seguro de incendio para su casa
        \begin{itemize}
            \item La casa puede no incendiarse y el individuo no pierde nada: Evento Bueno con probabilidad $p$
            \item La casa puede incendiarse y el individuo sufre una pérdida de tamaño $L$: Evento Malo con probabilidad $(1-p)$ 
        \end{itemize}
        \item La probabilidad del evento bueno depende en parte de alguna acción del individuo, vamos a decir que depende del esfuerzo del individuo: $p(e)$
        \begin{itemize}
            \item Por ejemplo: el individuo esta alerta a no dejar electrodomésticos enchufados, ni hornallas encendidas, le hace mantenimiento al hagor, etc.
            \item Cuanto más alto el esfuerzo, menos la probabilidad del incidente. Pero\dots
        \end{itemize}
        \item La clave es entender que quien ofrece el seguro no puede ver esta acción o esfuerzo
    \end{itemize}
\end{frame}

\begin{frame}{Riesgo moral y el colapso del mercado de seguros}
    \begin{itemize}
        \item Si la compañía aseguradora ofrece una cobertura de valor $C$ a un precio $\pi C$ (el precio depende de la cobertura) \vspace{2mm}
        \item En el escenario bueno, la utilidad para el individuo es
        \begin{equation*}
            U_B = y - \pi C
        \end{equation*}
        \item Si se produce el evento malo, su utilidad para el individuo es
        \begin{equation*}
            U_M = y - L - \pi C + C
        \end{equation*}
        \item ¿Qué $\pi$ (precio) podría cobrar la compañía de seguros?
        \begin{itemize}
        \item La ecuación de ganancias de las aseguradoras es
        \begin{equation*}
            \pi C - (1-p) C
        \end{equation*}
        \item Si esta ganancia la hacemos $0$ el (menor) porcentaje que puede cobrar la compañía es $\pi=(1-p)$
        \end{itemize}
    \end{itemize}
\end{frame}

\begin{frame}{Riesgo moral y el colapso del mercado de seguros}
    \begin{itemize}
        \item Si el individuo compra una cobertura de $C=L$, es decir, se asegura totalmente:
        \begin{itemize}
        \item En el escenario bueno, la utilidad para el individuo es 
        $U_B = y - \pi L$
        \item Si se produce el evento malo, su utilidad para el individuo es 
        $U_M = y - L - \pi L + L = y - \pi L$ 
        \end{itemize}
        \item Le es indiferente si se produce el evento bueno o el malo 
        \item Pero entonces $e=0$, es decir, no va a esforzarse por cuidar la casa, y la probabilidad del evento malo va a ser más alta.
        \item Si el individuo no hace nada el siniestro ocurre con probabilidad $(1-p)=1$, y en este caso, $p=0$ y $\pi=1$
        \item La utilidad para el individuo de comprar seguro es $y - L$, que resulta lo mismo que no comprar seguro $y - () L$
        \item ¡Es decir que el mercado asegurador desaparece!
    \end{itemize}   
\end{frame}


\begin{frame}{Discusiones}
    \begin{itemize}
        \item ¿Por qué pierden valor los autos al salir de la concesionaria?
        \item Políticas de deducibles.
        \item Obama care.
    \end{itemize}
\end{frame}

\end{document}


\begin{frame}{Teoría de los contratos}
    \begin{itemize}
        \item Dijimos que los economistas trabajan para mejorar la asignación
      \item Muchas veces para eso hay que tener "buenas reglas de juego"
      \item Los contratos establecen las reglas de juego de ciertas interacciones. 
      \item Libertad para definir un contrato - Ejecución del contrato
      \item En esta clase vamos a aprender a hacer un contrato
      \item Para ello vamos a analizar el problema del principal-agente
  \end{itemize}
\end{frame}


\begin{frame}{El problema del principal-agente}
    \begin{itemize}
        \item El Principal contrata a alguien para hacer algo
        \item El Agente es la persona contratada para hacer la tarea
        \item El agente toma decisiones que afectan al principal, pero esta motivado a seguir su propio interés - P.ej., votantes-políticos, empleador-empleado, accionista-CEO, pacientes-        \item Usualmente el problema de P-A surge debido a asimetrías de información:
\begin{itemize}
\item Es costoso para el principal ver lo que el agente hace
\item El agente sabe algo que el principal no
\item Lo útil para el principal es costoso para el agente
\end{itemize}
\item Este es el problema que el contrato tiene que resolver.
\end{itemize}
\end{frame}

\begin{frame}{El problema del principal-agente II}
    \begin{itemize}
        \item Empleador y empleado
        \begin{itemize}
            \item El empleador contrata a un trabajo y le ofrece un contrato
\item El empleado se esfuerza o no (no visible)
\item Empleador le paga en función de lo producido

        \end{itemize}
        
\item Seguros del auto
    \begin{itemize}
            \item La compañía de seguro ofrece una póliza con deducibles
            \item El conductor luego elije como manejar
            \item La compañía paga los accidentes
        \end{itemize}
        
\item  Accionistas y su CEO
    \begin{itemize}
\item Accionistas le ofrecen al CEO un contrato con acciones
\item CEO realiza su tarea
\item CEO se le paga en función del precio de la acción
        \end{itemize}
   \end{itemize}
\end{frame}

\begin{frame}{Diseñando el contrato}
\begin{itemize}
    \item 1ro. El empleador ofrece un contrato
    \item 2d En función del contrato pensamos la respuesta óptima del empleado
    \item 3ro Dada la respuesta óptima vemos que contrato conviene
    \end{itemize}
\end{frame}


\begin{frame}{Diseñando el contrato}
    \begin{itemize}
        \item La  función de producción $y = a$, donde $y$ es el producto y $a$ está determinado por la acción (esfuerzo) del empleado.
        \item Para el empleado  ejecutar $a$ tiene un costo $C(a)=\frac{a^2}{2}$
        \item El empleador ofrece $w=c+by$
        \item Tenemos que encontrar los  $c$ y $b$ óptimos para el empleador 
    \end{itemize}
\end{frame}

\begin{frame}{Diseñando el contrato}
    \begin{itemize}
        \item la utilidad para el empleado es $w-C(a)$.
        \item La utilidad para el empleador es $y-w$.
    \end{itemize}
\end{frame}

\begin{frame}{Diseñando un contrato III}
    \begin{figure} [H]
\centering
\begin{tikzpicture}[scale=0.6]
\draw[thick,<->] (0,10) node[above]{$C(a)$}--(0,0)--(10,0) node[right]{$a$};
\draw[thick, gray,dashed] (3.65,0)--(3.65,3.6);
\draw [semithick] (0,0)--(8.5,8.5)node[right]{\footnotesize $w=c+ba$} ;
\draw [semithick] (2.75,1.05)--(4.75,3.05) ;
\draw[semithick] (0,0)..controls (3,1) and (5,2) .. (6.5,8.5) node[above]{\footnotesize $C(a)$};
\node[below] at (3.65,0) {\footnotesize $b$};
\end{tikzpicture}
\label{fig:24.1}
\end{figure} 
\end{frame}

 
\begin{frame}{Diseñando un Contrato IV}
 \begin{itemize}
      \item En que punto la pendiente de la curva de esfuerzo es $b$?

 
\begin{equation}
        \frac{C(a+\Delta)-C(a)}{\Delta}=\frac{1}{2}\frac{a^2+2a\Delta+\Delta^2-a^2}{\Delta} = a   
       \end{equation} 
       \item La pendiente de $C(a)$ en cada punto de la curva es igual a $a$!
       \item Es decir la pendiente es $b$ cuando el esfuerzo es $b$.
\end{itemize}
\end{frame}

\begin{frame}{Diseñando un contrato V}
    \begin{itemize}
        \item Dada esto, la ganancia para el empleador es 
        
        \begin{equation}
         y-w= a-c-ba= b-c-bb= b(1-b)-c.  
        \end{equation}
        \item El empleador elegirá el $b$ que maximice su ganancia.
        \item el término $b(1-b)$ tiene su máximo en $b=\frac{1}{2}$
        \item Y la ganancia $b-c-bb=\frac{1}{2}-\frac{1}{4}-c=\frac{1}{4}-c$. 
        
    \end{itemize}
\end{frame}


\begin{frame}{Diseñando un contrato VI}
    \begin{itemize}
        \item Pero en realidad tenemos que maximizar respecto de $b$ y $c$.
        \item La utilidad del empleado es 
        \begin{equation}
         w-C(a)= c+ bb- \frac{b^2}{2}=c+ \frac{1}{4}-\frac{1}{2}\frac{1}{4}=c+\frac{1}{8}.
        \end{equation}
\item el menor $c$ posible que puede ofrecer el empleador es $-\frac{1}{8}$.
\item La utilidad para el empleado es cero y para el empleador es $\frac{3}{8}$.
\end{itemize}
\end{frame}


 
\begin{frame}{Diseñando un contrato VII}
    \begin{itemize}
        \item Supongamos ahora $b=1$
        \item Ahora $b=a=y=1$
        \item la ganancia para el empleador sería $1-c-1=-c$
        \item la ganancia para el empleado sería $c+1-\frac{1}{2}=c+1/2$
        \item en este caso, el empleador podría llegar a seleccionar $c=-\frac{1}{2}$
        \item Esto es todavia mejor! 
        \item Este es un contrato típico para un CEO o en Wall Street. 
    \end{itemize}
\end{frame}

\begin{frame}{Diseñando un contrato - Discusión}
    \begin{itemize}
        \item Caso parabrisas 
        \item Docentes universitarios - Tareas multidimensionales
        \item Innovación disruptiva - Caso de Capecchi 
        \item Tareas colectivas - Software
        \item Factores aleatorios - Agricultura
        \item Caso Lincoln Electrics
        \item Rol de las preocupaciones profesionales
        \item Mozo/a
    \end{itemize}
\end{frame}

