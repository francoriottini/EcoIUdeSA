\documentclass{beamer}
\usepackage{amsmath}
\usepackage[english]{babel} %set language; note: after changing this, you need to delete all auxiliary files to recompile
\usepackage[utf8]{inputenc} %define file encoding; latin1 is the other often used option
\usepackage{csquotes} % provides context sensitive quotation facilities
\usepackage{graphicx} %allows for inserting figures
\usepackage{booktabs} % for table formatting without vertical lines
\usepackage{textcomp} % allow for example using the Euro sign with \texteuro
\usepackage{stackengine}
\usepackage{wasysym}
\usepackage{tikzsymbols}
\usepackage{textcomp}
% ELIMINAR COMANDOS DE NAVEGACION%%%%%%%%%%%
\setbeamertemplate{navigation symbols}

%\newcommand{\bubblethis}[2]{
 %       \tikz[remember picture,baseline]{\node[anchor=base,inner sep=0,outer sep=0]%
 %       (#1) {\underline{#1}};\node[overlay,cloud callout,callout relative pointer={(0.2cm,-0.7cm)},%
 %       aspect=2.5,fill=yellow!90] at ($(#1.north)+(-0.5cm,1.6cm)$) {#2};}%
 %   }%
%\tikzset{face/.style={shape=circle,minimum size=4ex,shading=radial,outer sep=0pt,
 %       inner color=white!50!yellow,outer color= yellow!70!orange}}

% Some commands to make the code easier
\newcommand{\emoticon}[1][]{%
  \node[face,#1] (emoticon) {};
  %% The eyes are fixed.
  \draw[fill=white] (-1ex,0ex) ..controls (-0.5ex,0.2ex)and(0.5ex,0.2ex)..
        (1ex,0.0ex) ..controls ( 1.5ex,1.5ex)and( 0.2ex,1.7ex)..
        (0ex,0.4ex) ..controls (-0.2ex,1.7ex)and(-1.5ex,1.5ex)..
        (-1ex,0ex)--cycle;}
\newcommand{\pupils}{
  %% standard pupils
  \fill[shift={(0.5ex,0.5ex)},rotate=80] 
       (0,0) ellipse (0.3ex and 0.15ex);
  \fill[shift={(-0.5ex,0.5ex)},rotate=100] 
       (0,0) ellipse (0.3ex and 0.15ex);}

\newcommand{\emoticonname}[1]{
  \node[below=1ex of emoticon,font=\footnotesize,
        minimum width=4cm]{#1};}
\usepackage{scalerel}
\usetikzlibrary{positioning}
\usepackage{xcolor,amssymb}
\newcommand\dangersignb[1][2ex]{%
  \scaleto{\stackengine{0.3pt}{\scalebox{1.1}[.9]{%
  \color{red}$\blacktriangle$}}{\tiny\bfseries !}{O}{c}{F}{F}{L}}{#1}%
}
\newcommand\dangersignw[1][2ex]{%
  \scaleto{\stackengine{0.3pt}{\scalebox{1.1}[.9]{%
  \color{red}$\blacktriangle$}}{\color{white}\tiny\bfseries !}{O}{c}{F}{F}{L}}{#1}%
}
\usepackage{fontawesome} % Social Icons
\usepackage{epstopdf} % allow embedding eps-figures
\usepackage{tikz} % allows drawing figures
\usepackage{amsmath,amssymb,amsthm} %advanced math facilities
\usepackage{lmodern} %uses font that support italic and bold at the same time
\usepackage{hyperref}
\usepackage{tikz}
\hypersetup{
    colorlinks=true,
    linkcolor=blue,
    filecolor=magenta,      
    urlcolor=blue,
}
\usepackage{tcolorbox}
%add citation management using BibLaTeX
\usepackage[citestyle=authoryear-comp, %define style for citations
    bibstyle=authoryear-comp, %define style for bibliography
    maxbibnames=10, %maximum number of authors displayed in bibliography
    minbibnames=1, %minimum number of authors displayed in bibliography
    maxcitenames=3, %maximum number of authors displayed in citations before using et al.
    minnames=1, %maximum number of authors displayed in citations before using et al.
    datezeros=false, % do not print dates with leading zeros
    date=long, %use long formats for dates
    isbn=false,% show no ISBNs in bibliography (applies only if not a mandatory field)
    url=false,% show no urls in bibliography (applies only if not a mandatory field)
    doi=false, % show no dois in bibliography (applies only if not a mandatory field)
    eprint=false, %show no eprint-field in bibliography (applies only if not a mandatory field)
    backend=biber %use biber as the backend; backend=bibtex is less powerful, but easier to install
    ]{biblatex}
\addbibresource{../mybibfile.bib} %define bib-file located one folder higher


\usefonttheme[onlymath]{serif} %set math font to serif ones

\definecolor{beamerblue}{rgb}{0.2,0.2,0.7} %define beamerblue color for later use

%%% defines highlight command to set text blue
\newcommand{\highlight}[1]{{\color{blue}{#1}}}


%%%%%%% commands defining backup slides so that frame numbering is correct

\newcommand{\backupbegin}{
   \newcounter{framenumberappendix}
   \setcounter{framenumberappendix}{\value{framenumber}}
}
\newcommand{\backupend}{
   \addtocounter{framenumberappendix}{-\value{framenumber}}
   \addtocounter{framenumber}{\value{framenumberappendix}}
}

%%%% end of defining backup slides

%Specify figure caption, see also http://tex.stackexchange.com/questions/155738/caption-package-not-working-with-beamer
\setbeamertemplate{caption}{\insertcaption} %redefines caption to remove label "Figure".
%\setbeamerfont{caption}{size=\scriptsize,shape=\itshape,series=\bfseries} %sets figure  caption bold and italic and makes it smaller


\usetheme{Boadilla}

%set options of hyperref package
\hypersetup{
    bookmarksnumbered=true, %put section numbers in bookmarks
    naturalnames=true, %use LATEX-computed names for links
    citebordercolor={1 1 1}, %color of border around cites, here: white, i.e. invisible
    linkbordercolor={1 1 1}, %color of border around links, here: white, i.e. invisible
    colorlinks=true, %color links
    anchorcolor=black, %set color of anchors
    linkcolor=beamerblue, %set link color to beamer blue
    citecolor=blue, %set cite color to beamer blue
    pdfpagemode=UseThumbs, %set default mode of PDF display
    breaklinks=true, %break long links
    pdfstartpage=1 %start at first page
    }


% --------------------
% Overall information
% --------------------
\title[Principios de Economía]{Principios de Economía}
\date{}
\author[Ertola]{Gaby Ertola }
\vspace{0.4cm}
\institute[]{Universidad de San Andrés \\
2023} 

\begin{document}

\begin{frame}
\titlepage
\centering
\includegraphics[scale=0.25]{Slides Principios de Economia/Figures/logoUDESA.jpg} 
\end{frame}

\begin{frame}
\frametitle{Profesores y mails}
\begin{itemize}
    \item Gaby (gertolanavajas@udesa.edu.ar) \vspace{2mm}
    \item Victoria (rosinom@udesa.edu.ar) \vspace{2mm}
    \item Franco (friottinidepetris@udesa.edu.ar)
\end{itemize}
\end{frame}

\begin{frame}
\frametitle{Clases de consulta}
\begin{itemize}
    \item Todos tenemos horarios de consulta semanales \vspace{2mm}
    \item Se publican en la página principal del Campus Virtual \vspace{2mm}
    \item Los horarios pueden cambiar \vspace{2mm}
    \item ¡APROVECHEN LAS CONSULTAS DESDE EL COMIENZO!
\end{itemize}
\end{frame}

\begin{frame}
\frametitle{Modalidad de trabajo}
\begin{itemize}
    \item ¡Este curso implica mucho trabajo! Pero...  \vspace{2mm}
    \item ¡Vamos a trabajar en temas muy interesantes y útiles!  \vspace{2mm}
    \item 26 clases magistrales a lo largo del semestre 
\begin{itemize}
        \item 13 antes del receso de parciales, 13 en la segunda parte
        \item ¿Cuándo? Lunes y miércoles de 9 a 10:40
        \end{itemize}
    \item 13 tutoriales, una vez por semana, empezando la semana que viene 
\end{itemize}
\end{frame}

\begin{frame}
\frametitle{Es importante comenzar a estudiar desde el primer día}
\begin{itemize}
    \item Este curso tiene \textbf{EXAMEN FINAL}... \vspace{2mm}
    \item ... pero contiene un alto número de puntos de evaluación a lo largo del semestre:
        \begin{itemize}
            \item 10 pop-quizzes
            \item 13 tutoriales (con distintas evaluaciones dentro) y, al menos, 2 trabajos prácticos
            \item 1 examen parcial escrito presencial en el receso intermedio (fin de abril)
        \end{itemize} \vspace{2mm}
    \item que nos permite tener una proxy del desempeño del alumno: la ``nota umbral''!!!
    \end{itemize}
\end{frame}

\begin{frame}
\frametitle{Nota ``umbral''}
\begin{itemize}
    \item La nota ``umbral'' es un indicador del trabajo durante el semestre que \textbf{puede dar acceso a ``privilegios'' de evaluación} \vspace{2mm}
    \item ¿Cuenta todo? NO!
        \begin{itemize}
            \item Los 8 mejores quizzes
            \item Las mejores 9 ejercitaciones de tutoriales y los 2 trabajos prácticos
            \item La nota del primer examen parcial
        \end{itemize} \vspace{2mm}
    \item ¿Cómo la calculo? Haciendo un promedio de las notas que cuentan:
    \end{itemize}
    \begin{center}
    $N_{umbral}=1/3N_{quizzes}+1/3N_{tutoriales}+1/3N_{parcial 1}$    
    \end{center}
    
\end{frame}

\begin{frame}
\frametitle{Nota "umbral"}
\begin{itemize}
    \item La nota umbral se define el último día de clases, es decir, el 17 de noviembre    
    \begin{itemize} \vspace{2mm}
            \item Si la nota umbral es igual o mayor que 6 ($N_{umbral} \geq 6$) entonces acceden al Segundo Parcial
            \item Si la nota umbral es menor que 6 ($N_{umbral} < 6$) entonces deben rendir Examen Final      \end{itemize} \vspace{2mm}
            
    \item ¿Cómo son esos exámenes? Todo está en el programa
            \end{itemize}
\end{frame}

\begin{frame}
\frametitle{¿Quienes rinden Examen Final?} 
\begin{itemize}
    \item Aquellos que desaprobaron el examen parcial \vspace{2mm}
    \item Aquellos que, habiendo aprobado el parcial, tienen una nota umbral menor a 6 \vspace{2mm}
    \item Está contemplada la posibilidad de dar el final para aquellos que puedan acceder al segundo parcial pero deseen rendir el final para 'mejorar' la nota del parcial. Quien quiera hacer esto, debe confirmar la decisión vía email a los profesores antes del 17 de noviembre
\end{itemize}
\end{frame}

\begin{frame}
\frametitle{Nota del curso}
\small
\begin{itemize}
    \item Para los que hayan accedido al segundo parcial \\
\end{itemize}
\begin{center}
    {$N_{curso}=0,1N_{quizzes}+0,2N_{tutorial}+0,35N_{parcial 1}+0,35N_{parcial 2}$}
\end{center} \vspace{2mm}
\begin{itemize}
    \item Para los que hayan accedido al final
\end{itemize} 
\begin{center}
    {$N_{curso}=0,1N_{quizzes}+0,2N_{tutorial}+0,7N_{final}$}
\end{center} \vspace{2mm}
\begin{itemize}    
    \item Deben notar que, si van a rendir final, la nota del parcial ¡NO cuenta!
\end{itemize}
\end{frame}

\begin{frame}
\frametitle{Para aprobar este curso}
Para pasar este curso es estrictamente necesario obtener al menos 4 puntos en:
\vspace{2mm}
\begin{itemize}
    \item Ambos parciales ($N_{parcial 1} \geq 4$ y $N_{parcial 2} \geq 4$) o el examen final sin redondeo ($N_{final} \geq 4$)
\end{itemize}
\centering y
\vspace{2mm}
\begin{itemize}
    \item La nota del curso ($N_{curso} \geq 4$)
\end{itemize}
\end{frame}

\begin{frame}
\frametitle{Recuperatorio}
\begin{itemize}
    \item Es igual que un examen final \vspace{2mm}
    \item Sólo para los estudiantes que hayan tenido una nota ‘umbral’ igual o mayor a 6
\end{itemize}
    \begin{center}
     $N_{umbral}=1/3N_{quizzes}+1/3N_{tutorial}+1/3N_{parcial 1} \geq 6$   
    \end{center}
\begin{itemize} \vspace{2mm}
    \item El recuperatorio es como un examen final pero la nota del curso será como máximo 6:
\end{itemize}
\end{frame}

\begin{frame}
\frametitle{Materiales: TODO en el Campus Virtual}
\begin{itemize}
    \item Capítulos del libro \vspace{2mm}  \\
    Ertola y Struzenegger [2022]:  \textit{Principios de Economía} \vspace{2mm} 
    \item Slides de las magistrales (antes de la clase) \vspace{2mm}
    \item Ejercitaciones para las tutoriales \vspace{2mm}
    \item Recursos adicionales \vspace{2mm}
    \item Grabaciones \vspace{2mm}
\end{itemize}
\end{frame}

\begin{frame}
\begin{center}
    \Huge ¿Preguntas?
\end{center}
\end{frame}

\end{document}

