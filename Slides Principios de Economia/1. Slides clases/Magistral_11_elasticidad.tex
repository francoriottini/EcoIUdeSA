\documentclass{beamer}
\usepackage{amsmath}
\usepackage[english]{babel} %set language; note: after changing this, you need to delete all auxiliary files to recompile
\usepackage[utf8]{inputenc} %define file encoding; latin1 is the other often used option
\usepackage{csquotes} % provides context sensitive quotation facilities
\usepackage{graphicx} %allows for inserting figures
\usepackage{booktabs} % for table formatting without vertical lines
\usepackage{textcomp} % allow for example using the Euro sign with \texteuro
\usepackage{stackengine}
\usepackage{wasysym}
\usepackage{tikzsymbols}
\usepackage{textcomp}
% ELIMINAR COMANDOS DE NAVEGACION%%%%%%%%%%%
\setbeamertemplate{navigation symbols}

%\newcommand{\bubblethis}[2]{
 %       \tikz[remember picture,baseline]{\node[anchor=base,inner sep=0,outer sep=0]%
 %       (#1) {\underline{#1}};\node[overlay,cloud callout,callout relative pointer={(0.2cm,-0.7cm)},%
 %       aspect=2.5,fill=yellow!90] at ($(#1.north)+(-0.5cm,1.6cm)$) {#2};}%
 %   }%
%\tikzset{face/.style={shape=circle,minimum size=4ex,shading=radial,outer sep=0pt,
 %       inner color=white!50!yellow,outer color= yellow!70!orange}}

%% Some commands to make the code easier
\newcommand{\emoticon}[1][]{%
  \node[face,#1] (emoticon) {};
  %% The eyes are fixed.
  \draw[fill=white] (-1ex,0ex) ..controls (-0.5ex,0.2ex)and(0.5ex,0.2ex)..
        (1ex,0.0ex) ..controls ( 1.5ex,1.5ex)and( 0.2ex,1.7ex)..
        (0ex,0.4ex) ..controls (-0.2ex,1.7ex)and(-1.5ex,1.5ex)..
        (-1ex,0ex)--cycle;}
\newcommand{\pupils}{
  %% standard pupils
  \fill[shift={(0.5ex,0.5ex)},rotate=80] 
       (0,0) ellipse (0.3ex and 0.15ex);
  \fill[shift={(-0.5ex,0.5ex)},rotate=100] 
       (0,0) ellipse (0.3ex and 0.15ex);}

\newcommand{\emoticonname}[1]{
  \node[below=1ex of emoticon,font=\footnotesize,
        minimum width=4cm]{#1};}
\usepackage{scalerel}
\usetikzlibrary{positioning}
\usepackage{xcolor,amssymb}
\newcommand\dangersignb[1][2ex]{%
  \scaleto{\stackengine{0.3pt}{\scalebox{1.1}[.9]{%
  \color{red}$\blacktriangle$}}{\tiny\bfseries !}{O}{c}{F}{F}{L}}{#1}%
}
\newcommand\dangersignw[1][2ex]{%
  \scaleto{\stackengine{0.3pt}{\scalebox{1.1}[.9]{%
  \color{red}$\blacktriangle$}}{\color{white}\tiny\bfseries !}{O}{c}{F}{F}{L}}{#1}%
}
\usepackage{fontawesome} % Social Icons
\usepackage{epstopdf} % allow embedding eps-figures
\usepackage{tikz} % allows drawing figures
\usepackage{amsmath,amssymb,amsthm} %advanced math facilities
\usepackage{lmodern} %uses font that support italic and bold at the same time

\usepackage{tikz}

\usepackage{tcolorbox}

\usefonttheme[onlymath]{serif} %set math font to serif ones

\definecolor{beamerblue}{rgb}{0.2,0.2,0.7} %define beamerblue color for later use

%%% defines highlight command to set text blue
\newcommand{\highlight}[1]{{\color{blue}{#1}}}


%%%%%%% commands defining backup slides so that frame numbering is correct

\newcommand{\backupbegin}{
   \newcounter{framenumberappendix}
   \setcounter{framenumberappendix}{\value{framenumber}}
}
\newcommand{\backupend}{
   \addtocounter{framenumberappendix}{-\value{framenumber}}
   \addtocounter{framenumber}{\value{framenumberappendix}}
}

%%%% end of defining backup slides

%Specify figure caption, see also http://tex.stackexchange.com/questions/155738/caption-package-not-working-with-beamer
\setbeamertemplate{caption}{\insertcaption} %redefines caption to remove label "Figure".
%\setbeamerfont{caption}{size=\scriptsize,shape=\itshape,series=\bfseries} %sets figure  caption bold and italic and makes it smaller


\usetheme{Boadilla}


% --------------------
% Overall information
% --------------------
\title[Economía I]{Economía I \vspace{4mm}
\\ Magistral 11: Elasticidad}
\date{}
\author[Riottini]{Riottini Franco}
\vspace{0.4cm}
\institute[]{Universidad de San Andrés} 

\begin{document}

\begin{frame}
\titlepage
\centering
\includegraphics[scale=0.2]{../Figures/logoUDESA.jpg} 
\end{frame}

%\begin{frame}{Elasticidad precio de rutas áereas}
%\centering
%\includegraphics[scale=0.55]{../Figures/ElasticidadVuelos.png}
%\end{frame}

\begin{frame}
\frametitle{Sensibilidad de la demanda}
\begin{itemize}
    \item La pendiente de la curva de demanda tiene información relevante para las empresas\vspace{2mm}
    \begin{itemize}
        \item La pendiente muestra la relación costo-beneficio que la firma enfrenta entre precio y cantidad.\vspace{4mm}
     \end{itemize}
    \item ¿Nos ayuda a pensar qué tan sensible es la demanda ante cambios en los precios?\vspace{2mm}
    \begin{itemize}
        \item En parte, si, pero para pensar en esto usamos el concepto de \textbf{elasticidad-precio}: \\\vspace{2mm}
      \item Intuitivamente, se refiere al grado de reacción de los consumidores a cambios en el precio del producto 
    \end{itemize}
    \end{itemize}
\end{frame}

\begin{frame}
\frametitle{Elasticidad precio de la demanda}
\begin{itemize}
    \item A diferencia de la pendiente, la elasticidad precio mira cambios porcentuales:
    \begin{center}
    \includegraphics[scale=0.7]{../Figures/Tema_06.43_elasticidadformula.png}
    \end{center}
    \end{itemize}
\end{frame}

\begin{frame}
\frametitle{Elasticidad precio de la demanda}
\begin{itemize}
    \item Medida de sensibilidad: ¿cuál es el cambio \% en la cantidad demandada ante un cambio de 1\% en el precio? \\\vspace{2mm}
    \begin{itemize}
        \item Como la demanda cae ante un aumento en el precio, se suele cambiar el signo para que la relación nos de positiva (esto facilita la interpretación)\vspace{4mm}
    \end{itemize}
    \item ¿Qué tan elástica? \\\vspace{2mm}
    - Si $e > 1$ decimos que la demanda es elástica \\
    - Si $e = 1$ decimos que la demanda es unitaria \\
    - Si $e < 1$ decimos que la demanda es inelástica
    \end{itemize}
\end{frame}

% \begin{frame}
% \frametitle{Determinantes de la elasticidad precio de la demanda}
% \begin{itemize}
%     \item Disponibilidad de sustitutos cercanos: bienes con sustitutos cercanos tienden a tener demandas más elásticas.\vspace{4mm}
%  \item Necesidades vs Lujos: necesidades tienden a tener demandas inelásticas, mientras que los lujos demandas elásticas.\vspace{4mm}
% \item Definición de mercado: Helado. Helado bombon escoces.\vspace{4mm}
% \item Horizonte de tiempo: Gasolina. Cae poco la cantidad demandada en el CP y más en el LP
%     \end{itemize}
% \end{frame}

% \begin{frame}
% \frametitle{Elasticidad y pendiente}
% \begin{itemize}
%     \item Ambos conceptos están relacionados
%     \begin{center}
%     \includegraphics[scale=0.5]{../Figures/Tema_06.44_elasticidadpendiente.png}
%     \end{center}
%         \begin{itemize}
%         \item La pendiente forma parte del concepto de elasticidad.\vspace{2mm}
%         \item Una curva de demanda muy empinada es relativamente inelástica, y una bastante plana es elástica.\vspace{4mm}
%         \end{itemize}
%     \item ¡Pero no son lo mismo!\vspace{2mm}
%         \begin{itemize}
%             \item Notar que la elasticidad puede cambiar a medida que nos movemos a lo largo de la curva de demanda, aun si la pendiente no lo hace.\vspace{2mm}
%             \item ¡Y viceversa!
%         \end{itemize}
%     \end{itemize}
% \end{frame}

% \begin{frame}
% \frametitle{Elasticidad constante}
% \includegraphics[scale=0.6]{../Figures/Tema_06.45_elasticidad.png}
% \end{frame}

% \begin{frame}
% \frametitle{Pendiente constante}
% \includegraphics[scale=0.6]{../Figures/Tema_06.46_elasticidad2.png}
% \end{frame}

% \begin{frame}
% \frametitle{¿Cómo son los excedentes con una demanda elástica?}
% \includegraphics[scale=0.6]{../Figures/Tema_07.24_equilibrioyexcedente2.jpg}
% \end{frame}

% \begin{frame}
% \frametitle{¿Cómo son los excedentes con una demanda inelástica?}
% \includegraphics[scale=0.6]{../Figures/Tema_07.25_equilibrioyexcedente3.jpg}
% \end{frame}

% \begin{frame}
% \frametitle{¿Cómo son los excedentes con una oferta elástica?}
% \includegraphics[scale=0.6]{../Figures/Tema_07.26_equilibrioyexcedente4.jpg}
% \end{frame}

% \begin{frame}
% \frametitle{¿Cómo son los excedentes con una oferta inelástica?}
% \includegraphics[scale=0.55]{../Figures/Tema_07.25_equilibrioyexcedente3.jpg}
% \end{frame}

% \begin{frame}
% \frametitle{Ingreso total (pxq) y elasticidad}
% \begin{itemize}
%     \item El impacto de un cambio de precio en los ingresos totales, depende de la elasticidad de la demanda.
%     \item Si la demanda es inelástica un incremento en el precio provoca un decremento en la cantidad demandada proporcionalmente más pequeño, por lo cual los ingresos totales se incrementan.
%     \item Si la curva de la demanda es elástica, un incremento en el precio provoca una disminución en la cantidad demandada proporcionalmente más grande, por lo que los ingresos totales disminuyen.
%     \item Es decir:
%     \begin{itemize}
%         \item Cuando la demanda es inelástica, el precio y los ingresos totales se mueven en la misma dirección.
%         \item Cuando la demanda es elástica, el precio y los ingresos totales se mueven en direcciones opuestas.
%     \end{itemize}
% \end{itemize}
% \end{frame}

% \begin{frame}
% \frametitle{Elasticidad}
% \small Hay dos formas diferentes de estimar la elasticidad
% \centering
% \begin{block}{Elasticidad puntual}
% \begin{equation}
% \epsilon = -  \frac{\bigtriangleup \% Q}{\bigtriangleup \% P}
% \end{equation}
% \begin{equation}
% \epsilon = - \frac{\frac{\bigtriangleup Q}{Q}}{\frac{\bigtriangleup P}{P}}
% \end{equation}
% \begin{equation}
% \epsilon = - \frac{\bigtriangleup Q}{ \bigtriangleup P} \frac{P}{Q}
% \end{equation}
% \end{block}
% \end{frame}


% \begin{frame}
% \frametitle{Elasticidad}
% \small 
% \centering
% \begin{block}{Elasticidad de punto medio}
% \begin{equation} 
% \epsilon = -  \frac{\bigtriangleup \% Q}{\bigtriangleup \% P}
% \end{equation}
% \begin{equation}
% \epsilon = -  \frac{\frac{\bigtriangleup Q}{(Q_2+Q_1)/2}}{\frac{\bigtriangleup P}{(P_2+P_1)/2}}
% \end{equation}
% \begin{equation}
% \epsilon = - \frac{\frac{\bigtriangleup Q}{\text{Punto medio de Q}}}{\frac{\bigtriangleup P}{\text{Punto medio de P}}}
% \end{equation}
% \begin{equation}
% \epsilon = - \frac{\bigtriangleup Q}{ \bigtriangleup P} \frac{\text{Punto medio de P}}{\text{Punto medio de Q}}
% \end{equation}
% \end{block}
% \end{frame}


% \begin{frame}
% \frametitle{Otras elasticidades}
% \begin{itemize}
%     \item Elasticidad cruzada
%     \begin{itemize}
%         \item Mide la influencia del precio de un bien sobre el otro.
%         \item Si esta elasticidad es mayor a 0, los bienes son sustitutos, es decir, cuando se incrementa el precio de y aumenta la cantidad demandada de x; 
%         \item Si es menor a 0, los bienes son complementarios, es decir, si aumenta el precio de y baja la cantidad demandada de X . \vspace{4mm}
%     \end{itemize}
%     \item Elasticidad ingreso
%         \begin{itemize}
%         \item Si la elasticidad precio es mayor a 0, estamos en presencia de bienes normales
%         \item si es menor a 0, son bienes inferiores. 
%         \item A su vez, si la elasticidad ingreso es mayor a 1 es un bien de lujo.
%     \end{itemize}
% \end{itemize}
% \end{frame}

% \begin{frame}
% \frametitle{Un ejemplo....}
% \begin{itemize}
%     \item Pachi es una joven pastelera que inicio su propio negocio\vspace{2mm}
%     \item ¿Cual es la elasticidad de cada producto.....
%     \end{itemize}
% \centering
% \includegraphics[scale=0.6]{../Figures/Pachi.png}
% \end{frame}

% \begin{frame}
% \frametitle{El impacto de los impuestos}
% \begin{itemize}
%     \item ¿Qué es un impuesto?\vspace{2mm}
%     \begin{itemize}
%         \item Es un tributo generalmente establecido por el Estado 
%        \begin{itemize} 
%         - para financiar sus gastos
%         - para ‘guiar’ el comportamiento (por ejemplo en el caso de externalidades)\vspace{4mm}
%         \end{itemize}   
%     \end{itemize}
%     \item Existen diversos tipos:\vspace{2mm}
%     \begin{itemize}
%         \item Al consumo, al trabajo, al ingreso, a la propiedad, etc.
%     \end{itemize}\vspace{4mm}
%     \item Un impuesto aumentará el precio que los consumidores pagan sobre un bien...\vspace{4mm}
%     \item La elasticidad precio de la demanda tendrá influencia sobre el efecto del impuesto.
% \end{itemize}
% \end{frame}

% \begin{frame}
% \frametitle{Impuestos y eficiencia}
% \begin{itemize}
%     \item La introducción de impuestos aleja la economía del equilibrio competitivo\vspace{2mm}
%     \begin{itemize}
%         \item Los impuestos sobre oferentes/consumidores desplazan la curva de oferta/demanda porque el precio es más alto para cada cantidad
%         \item Al recaudar impuestos el Estado genera una pérdida de peso muerto.\vspace{4mm}
%     \end{itemize}
%     \item La recaudación se extrae del excedente de consumidores y productores
%     \begin{itemize}
%         \item La incidencia del impuesto depende de la elasticidad relativa de consumidores y productores
%         \item El grupo menos elástico lleva más de la carga fiscal
%     \end{itemize}
% \end{itemize}
% \end{frame}


% \begin{frame}
% \frametitle{Impuestos y elasticidad}
% \begin{itemize}
%     \item Un impuesto puede reducir mucho las ventas si su demanda es altamente elástica.\vspace{2mm}
%     \begin{itemize}
%         \item ¡Y eso puede ser lo que el gobierno intenta hacer! \\
%         - P.ej., impuestos sobre bienes ‘malos’ para la sociedad como el tabaco o el alcohol o por contaminar.\vspace{4mm}
%     \end{itemize}
%     \item Pero si un impuesto causa una importante caída en las ventas, también reduce los ingresos del impuesto.\vspace{4mm}
%     \item Si un gobierno que desea aumentar los ingresos a partir del impuesto, debería elegir gravar productos con demanda inelástica.\vspace{4mm}
%     \begin{itemize}
%         \item ¿Qué tipo de productos pueden tener una demanda de estas características?
%     \end{itemize}
% \end{itemize}
% \end{frame}

\end{document}

% \begin{frame}
% \frametitle{Impacto de los impuestos}
% \begin{itemize}
%     \item ¿Cuánto va a cambiar los impuestos el comportamiento de los individuos?
%     \begin{itemize}
%         \item ¿Qué tan grande va a ser la pérdida de peso muerto? \\
%         - ¿Porqué es relevante conocer la elasticidad de demanda? \\
%         - ¿Qué tipo de productos tienen demanda inelástica?
%     \end{itemize}
%     \item ¿Qué hace el gobierno con los recursos que recauda?
%     \item A veces el gobierno quiere cambiar el comportamiento
% \end{itemize}
% \end{frame}

% \begin{frame}
% \frametitle{Impacto de un impuesto a los vendedores}
% \includegraphics[scale=0.6]{../Figures/Tema_07.29_impuesto1.jpg}
% \end{frame}

% \begin{frame}
% \frametitle{Impacto de un impuesto a los vendedores}
% \includegraphics[scale=0.6]{../Figures/Tema_07.30_impuesto2.jpg}
% \end{frame}