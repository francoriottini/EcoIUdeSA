\documentclass{beamer}
\usepackage{amsmath}
\usepackage[english]{babel} %set language; note: after changing this, you need to delete all auxiliary files to recompile
\usepackage[utf8]{inputenc} %define file encoding; latin1 is the other often used option
\usepackage{csquotes} % provides context sensitive quotation facilities
\usepackage{graphicx} %allows for inserting figures
\usepackage{booktabs} % for table formatting without vertical lines
\usepackage{textcomp} % allow for example using the Euro sign with \texteuro
\usepackage{stackengine}
\usepackage{wasysym}
\usepackage{tikzsymbols}
\usepackage{textcomp}
\usepackage{ragged2e}
\usepackage{fontawesome} % Social Icons
\usepackage{epstopdf} % allow embedding eps-figures
\usepackage{tikz} % allows drawing figures
\usepackage{amsmath,amssymb,amsthm} %advanced math facilities
\usepackage{xcolor}

\usepackage{tikz}
\usepackage{tcolorbox}
\usepackage{hyperref}
% ELIMINAR COMANDOS DE NAVEGACION%%%%%%%%%%%
\setbeamertemplate{navigation symbols}

\newtcolorbox{boxA}{
    fontupper = \bf,
    boxrule = 1.5pt,
    colframe = black % frame color
}
\newtcolorbox{boxB}{
    boxrule = 1.5pt,
    colframe = blue!70!black,, % frame color
    colback = blue!7!white,
}

%\newcommand{\bubblethis}[2]{
 %       \tikz[remember picture,baseline]{\node[anchor=base,inner sep=0,outer sep=0]%
 %       (#1) {\underline{#1}};\node[overlay,cloud callout,callout relative pointer={(0.2cm,-0.7cm)},%
 %       aspect=2.5,fill=yellow!90] at ($(#1.north)+(-0.5cm,1.6cm)$) {#2};}%
 %   }%
%\tikzset{face/.style={shape=circle,minimum size=4ex,shading=radial,outer sep=0pt,
 %       inner color=white!50!yellow,outer color= yellow!70!orange}}

%% Some commands to make the code easier
\newcommand{\emoticon}[1][]{%
  \node[face,#1] (emoticon) {};
  %% The eyes are fixed.
  \draw[fill=white] (-1ex,0ex) ..controls (-0.5ex,0.2ex)and(0.5ex,0.2ex)..
        (1ex,0.0ex) ..controls ( 1.5ex,1.5ex)and( 0.2ex,1.7ex)..
        (0ex,0.4ex) ..controls (-0.2ex,1.7ex)and(-1.5ex,1.5ex)..
        (-1ex,0ex)--cycle;}
\newcommand{\pupils}{
  %% standard pupils
  \fill[shift={(0.5ex,0.5ex)},rotate=80] 
       (0,0) ellipse (0.3ex and 0.15ex);
  \fill[shift={(-0.5ex,0.5ex)},rotate=100] 
       (0,0) ellipse (0.3ex and 0.15ex);}

\newcommand{\emoticonname}[1]{
  \node[below=1ex of emoticon,font=\footnotesize,
        minimum width=4cm]{#1};}
\usepackage{scalerel}
\usetikzlibrary{positioning}
\usepackage{xcolor,amssymb}
\newcommand\dangersignb[1][2ex]{%
  \scaleto{\stackengine{0.3pt}{\scalebox{1.1}[.9]{%
  \color{red}$\blacktriangle$}}{\tiny\bfseries !}{O}{c}{F}{F}{L}}{#1}%
}
\newcommand\dangersignw[1][2ex]{%
  \scaleto{\stackengine{0.3pt}{\scalebox{1.1}[.9]{%
  \color{red}$\blacktriangle$}}{\color{white}\tiny\bfseries !}{O}{c}{F}{F}{L}}{#1}%
}
\usepackage{fontawesome} % Social Icons
\usepackage{epstopdf} % allow embedding eps-figures
\usepackage{tikz} % allows drawing figures
\usepackage{amsmath,amssymb,amsthm} %advanced math facilities
\usepackage{lmodern} %uses font that support italic and bold at the same time
\usepackage{hyperref}
\usepackage{tikz}
\hypersetup{
    colorlinks=true,
    linkcolor=blue,
    filecolor=magenta,      
    urlcolor=blue,
}
\usepackage{tcolorbox}
%add citation management using BibLaTeX
\usepackage[citestyle=authoryear-comp, %define style for citations
    bibstyle=authoryear-comp, %define style for bibliography
    maxbibnames=10, %maximum number of authors displayed in bibliography
    minbibnames=1, %minimum number of authors displayed in bibliography
    maxcitenames=3, %maximum number of authors displayed in citations before using et al.
    minnames=1, %maximum number of authors displayed in citations before using et al.
    datezeros=false, % do not print dates with leading zeros
    date=long, %use long formats for dates
    isbn=false,% show no ISBNs in bibliography (applies only if not a mandatory field)
    url=false,% show no urls in bibliography (applies only if not a mandatory field)
    doi=false, % show no dois in bibliography (applies only if not a mandatory field)
    eprint=false, %show no eprint-field in bibliography (applies only if not a mandatory field)
    backend=biber %use biber as the backend; backend=bibtex is less powerful, but easier to install
    ]{biblatex}
\addbibresource{../mybibfile.bib} %define bib-file located one folder higher


\usefonttheme[onlymath]{serif} %set math font to serif ones

\definecolor{beamerblue}{rgb}{0.2,0.2,0.7} %define beamerblue color for later use

%%% defines highlight command to set text blue
\newcommand{\highlight}[1]{{\color{blue}{#1}}}


%%%%%%% commands defining backup slides so that frame numbering is correct

\newcommand{\backupbegin}{
   \newcounter{framenumberappendix}
   \setcounter{framenumberappendix}{\value{framenumber}}
}
\newcommand{\backupend}{
   \addtocounter{framenumberappendix}{-\value{framenumber}}
   \addtocounter{framenumber}{\value{framenumberappendix}}
}

%%%% end of defining backup slides

%Specify figure caption, see also http://tex.stackexchange.com/questions/155738/caption-package-not-working-with-beamer
\setbeamertemplate{caption}{\insertcaption} %redefines caption to remove label "Figure".
%\setbeamerfont{caption}{size=\scriptsize,shape=\itshape,series=\bfseries} %sets figure  caption bold and italic and makes it smaller


\usetheme{Boadilla}

%set options of hyperref package
\hypersetup{
    bookmarksnumbered=true, %put section numbers in bookmarks
    naturalnames=true, %use LATEX-computed names for links
    citebordercolor={1 1 1}, %color of border around cites, here: white, i.e. invisible
    linkbordercolor={1 1 1}, %color of border around links, here: white, i.e. invisible
    colorlinks=true, %color links
    anchorcolor=black, %set color of anchors
    linkcolor=beamerblue, %set link color to beamer blue
    citecolor=blue, %set cite color to beamer blue
    pdfpagemode=UseThumbs, %set default mode of PDF display
    breaklinks=true, %break long links
    pdfstartpage=1 %start at first page
    }


% --------------------
% Overall information
% --------------------
\title[Principios de Economía]{Principios de Economía \vspace{4mm}
\\ Magistral 1}
\date{}
\author[Victoria Rosino]{Victoria Rosino}
\vspace{0.4cm}
\institute[]{Universidad de San Andrés} 

\begin{document}

\begin{frame}
    \vspace{0.4cm}
    \titlepage
    \centering
    \vspace{-0.5cm}
    \includegraphics[scale=0.3]{Slides Principios de Economia/Figures/udesa_logo.jpg} 
\end{frame}


%\begin{frame}
%\frametitle{Juego de libre asociación}
%\begin{itemize}
%    \item Ingresen a www.menti.com
%    \item Usen el código  6621 1327
%    \item Ingresen 3 palabras que asocien con economía
%\end{itemize} 
%    \begin{center}
%    \includegraphics[scale=0.2]{Slides Principios de Economia/Figures/qr.png}
%    \end{center}
%\end{frame}

\begin{frame}
\frametitle{¿Qu\'{e} es la economía?}
    \begin{itemize}
        \item Etimología de la palabra economía: del griego \textit{oikos} (casa, patrimonio) y \textit{nemein} (administrar). \vspace{1.5mm}
        \item Es una ciencia: tiene un objeto propio (la asignación de recursos escasos), un método (inductivo - deductivo) y un conjunto de teorías económicas capaces de explicar estos fenómenos. \vspace{1.5mm}
        \item Es una ciencia empírica: se puede contrastar con la realidad. \vspace{1.5mm}
        \item Es una ciencia social: estudia diversos aspectos de las sociedades. \vspace{1.5mm}
        \item NO es una ciencia exacta.
        \begin{center}
        \begin{boxA}
        La economía es una ciencia empírica y social que se ocupa de la manera en que se administran los recursos escasos  
        \end{boxA}
        \end{center}
    \end{itemize}
\end{frame}

\begin{frame}
\frametitle{¿Qu\'{e} hace un economista?}
    \centering 
    \includegraphics[scale=0.45]{Slides Principios de Economia/Figures/economist.jpeg}
    \\
\end{frame}


\begin{frame}
    \frametitle{¿Qu\'{e} hace un economista?}
    \begin{itemize}
        \item Busca asignar eficientemente recursos escasos \vspace{3mm} \\
        \item ¿Que quiere decir ``eficientemente''? \vspace{3mm} \\
        \item ¿Qué mecanismos tiene para realizar dicha asignación? 
        \begin{itemize}
            \item Sorteo?
            \item El que primero levanta la mano?
            \item El que más necesita?
            \item Designar alguien que decida?
            \begin{boxA}
            \begin{center}
                \textbf{O EL MERCADO!}
            \end{center}
            \end{boxA}
        \end{itemize}
    \end{itemize}
\end{frame}

\begin{frame}
    \frametitle{¿Cómo los puedo asignar?}
    El color de la persona indica la preferencia por el color de la taza. Supongamos que el mecanismo de asignación es un sorteo.
    \begin{center}
        \includegraphics[scale=0.8]{../Figures/Tazon.png}
    \end{center}
    \begin{itemize}
        \item ¿Es eficiente esta asignación?
        \item ¿Qué pasaría si el mecanismo de asignación fuera un planificador central? ¿A gran escala se puede?
        \item ¿Y un mercado? La gente podría cambiar el que quisiera por uno de su preferencia
    \end{itemize}
\end{frame}

\begin{frame}
\frametitle{¿Y ahora?}
    \begin{center}
        \includegraphics[scale=0.55]{../Figures/Tazon3.png}
    \end{center}
    \begin{itemize}
        \small
        \item La idea de eficiencia es muy relevante ya que implica un \textit{win-win}
        \begin{center}
            \begin{boxA} 
                \textbf{Una asignación es \textit{eficiente} cuando \textit{no} podemos mejorar el bienestar de alguien (o de varios) sin perjudicar a otros}
            \end{boxA}
        \end{center}
        \item Ojo: Eficiente no quiere decir justa...
        \item Eficiencia y equidad (distribución) son asuntos distintos para el economista
    \end{itemize}
\end{frame}

\begin{frame}
\frametitle{Con las figuritas pasa lo mismo...}
    \begin{center}
    \includegraphics[scale=0.3]{Slides Principios de Economia/Figures/eficiencia_ejemplo.jpg}
    \end{center}
\end{frame}


\begin{frame}
    \frametitle{¿Qu\'{e} aspectos estudia la economía?}
    \begin{itemize}
        \item Cómo tomamos decisiones  \vspace{2mm}
        \item Cómo interactuamos unos con otros (compradores y vendedores, empleados y empleadores, ciudadanos y servidores públicos, padres, hijos y familia) \vspace{2mm}
        \item Cómo interactuamos con nuestro entorno \vspace{2mm}
        \item Cómo estas cosas cambian en el tiempo  
    \end{itemize}
\end{frame}

\begin{frame}
    \frametitle{Entendiendo el mundo}
    Estudiar las causas y consecuencias de los problemas sociales es un reto \vspace{2mm}
    \begin{itemize}
        \item Responder algunas preguntas no es fácil...
        \begin{itemize}
            \item ¿Por qué algunos países son ricos y otros son pobres?
            \item ¿Por qué hay inflación en Argentina?
            \item ¿Las mujeres que se casan más tarde tienen niños más educados?
            \item ¿Causan los cinturones de seguridad más accidentes? \vspace{2mm}
            \end{itemize}
        \item ¡A veces es complicado comprender la pregunta!
    \end{itemize}
\end{frame}

\begin{frame}
\frametitle{¿Cómo hacer frente a estas preguntas?}
    \begin{itemize}
            \item Vamos a pensar utilizando modelos económicos... \vspace{2mm}
            \item y vamos confrontar esos modelos con la realidad
    \end{itemize} \vspace{2mm}
\begin{center}
    \includegraphics[scale=0.35]{Slides Principios de Economia/Figures/modelo.png}
\end{center}
\end{frame}

\begin{frame}
\frametitle{¿Qué es un modelo?}
\begin{itemize}
    \item ¡El mundo es muy complejo! Para entenderlo debemos simplificar la realidad \vspace{2mm} \\ $\Rightarrow$
    Todo comienza con una pregunta que deseamos responder \vspace{2mm}
    \item ¿Qué es un modelo? Es una simplificación de la realidad que construimos en base a \textbf{la pregunta de interés}  \vspace{2mm}
    \item  \href{https://www.youtube.com/watch?v=zwDA3GmcwJU}{``Del rigor de la ciencia''} 
    Jorge Luis Borges
\end{itemize}
\end{frame}

\begin{frame}
\frametitle{¿Cómo construimos modelos?}
    \begin{itemize}
        \item Planteamos una \textbf{hipótesis}: una suposición o afirmación que tenemos que comprobar con la evidencia disponible
        \begin{itemize}
        \item "La tierra es redonda"
        \item "Los salarios de las mujeres están por debajo de los de los hombres"
        \item "A mayor tiempo de estudio, mejores notas en los exámenes"
        \end{itemize}
        \item Validar una teoría: comparación de sus predicciones con la realidad.
        \item Nunca podemos afirmar que una hipótesis es cierta porque esté de acuerdo con los hechos.
        \item Si podemos negar la verdad de una hipótesis en base a ellos.
        \begin{center}
            \begin{boxA}
                \textbf{Una teoría estará bien corroborada, no cuando esté de acuerdo con un gran número de hechos, sino cuando seamos incapaces de encontrar hechos que la refuten. La teoría resiste y se consolida.}
            \end{boxA}
        \end{center}
    \end{itemize}
\end{frame}


\begin{frame}
    \frametitle{Modelos y supuestos}
    \begin{itemize}
        \item En un sentido estricto, todos los supuestos son irrealistas. No se puede hacer una descripción exacta de la realidad. \vspace{2mm}
        \item No hay que examinar el grado de realismo de los supuestos, sino su utilidad para entender la realidad. \vspace{2mm}
        \item Un \textbf{buen modelo}: 
    \begin{itemize}
        \item Predice con precisión: sus predicciones son consistentes con la evidencia \vspace{1mm}
        \item Mejora la comunicación. Nos ayuda a entender en qué estamos de acuerdo (y qué en desacuerdo) 
        \item Es útil. Podemos usarlo para encontrar formas de mejorar el funcionamiento de la economía
    \end{itemize}
    \end{itemize}
\end{frame}


\begin{frame}
    \frametitle{¿Por qué vamos a usar gráficos?}
    \begin{itemize}
        \item Los gráficos sirven para mostrar ideas \vspace{2mm}
        \item Los gráficos son útiles para ver si hay relación entre dos variables 
         \begin{itemize}
            \item ¡Pero tenemos que tener cuidado! Esas relaciones dan indicios de correlación... pero no de causalidad! \vspace{2mm}
        \end{itemize}
        \item Los gráficos (y sus expresiones matemáticas) pueden ser utilizados para describir una teoría
    \end{itemize} 
\end{frame}

\begin{frame}
    \frametitle{La correlación...}
    \begin{center}
        \includegraphics[scale=0.65]{Slides Principios de Economia/Figures/correlacion.png}
    \end{center}
\end{frame}

\begin{frame} \label{uno}
    \frametitle{Algunas explicaciones}
    \begin{itemize}
        \item OJO al interpretar relaciones entre variables porque \hyperlink{hdos}{correlación no implica causalidad} 
        \begin{boxB}
        \textbf{Correlación}: relación recíproca entre dos o más acciones, variables o fenómenos. \\
        \textbf{Causalidad}: relación causa-efecto. Hay causalidad si mover una variable x genera un cambio en la variable y.
        \end{boxB}
        \item Endogeneidad: ``todo tiene que ver con todo'' \vspace{1mm}
        \begin{itemize} 
         \item ¿En qué sentido va la correlación?
            \item Potenciales fuentes de confusión:
            \begin{itemize}
                \item {Omisión de variables}  
                \item {Simultaneidad} 
                \item {Error de medición en la variable explicativa}
            \end{itemize}
        \end{itemize}
    \end{itemize}
\end{frame}

%\begin{frame}
%\frametitle{Importancia de las Técnicas Cuantitativas}
%\begin{itemize}
%\item  Permiten evaluar hipótesis de teorías 
%\begin{itemize}
%\item ¿El desarrollo económico de un país impacta sobre su régimen político (dictadura/democracia)? 
%\item ¿Las organizaciones intergubernamentales promueven la paz?
%\item ¿Los timbreos tienen impacto sobre la decisión de voto de los electores?
%\end{itemize}
%\item Permiten evaluar políticas sociales y programas
%\begin{itemize}
%\item ¿La AUH tiene un impacto sobre la educación y la %salud de los beneficiarios?
%\item ¿Tienen los programas de crédito hipotecario del %Estado efectos sobre la violencia doméstica?
%\item ¿Poner cámaras de vigilancia en las esquinas impacta sobre el crimen en las calles?
%\end{itemize}
%\end{itemize}
%\end{frame}

\begin{frame} 
    \frametitle{¿Qué podemos hacer?}
    \begin{itemize}
        \item  En algunos casos se pueden diseñar experimentos o utilizar experimentos naturales (exógenos por definición) para detectar eventos que producen un cambio claro en una variable (algunos ejemplos \hyperlink{hdos}{acá})
         \item Usamos técnicas estadísticas que nos permitan "descubrir" la causalidad a partir los datos que tenemos. Para ello usamos algo que los economistas llaman "econometría". 
    \end{itemize}
\end{frame}
 
%\\begin{frame}  \label{unob}
%\\frametitle{¿Qué podemos hacer?}
%\\begin{itemize}
 %\   \item  \hyperlink{hseis}{Experimentos}
  %\  \item  \hyperlink{hsiete}{Experimentos naturales}
   %\ \item  \hyperlink{hocho}{Diferencias en Diferencias}
   %\ \item  \hyperlink{hnueve}{Regresión discontínua}
   %\ \item  \hyperlink{hdiez}{Control sintético}
%\\end{itemize}
%\\end{frame}

\begin{frame}
    \begin{center}
        \LARGE  \textbf{IMPORTANTE}  \\ \vspace{1cm}
        \Large 
        \begin{boxB}
        \centering Leer capítulos 2, 3 y 4 de esta clase
        \end{boxB}
        \vspace{0.5cm}
        \begin{boxB}
        Leer capítulos 5 y 6 para el viernes - HAY QUIZ!
        \end{boxB}
    \end{center}
\end{frame}

%\begin{frame} \label{hdos}
%\frametitle{Correlación versus Causalidad}
%\begin{center}
%    \includegraphics[scale=0.3]{Slides Principios de Economia/Figures/Introduccion_1.4.1_corrcaus.png}
%\end{center}
%Fuente: Google \\
%\hyperlink{uno}{Volver} 
%\end{frame}

%\begin{frame} \label{htres}
%\frametitle{Correlación versus Causalidad}
%\begin{center}
%    \includegraphics[scale=0.55]{Slides Principios de Economia/Figures/Introduccion_1.5_ovbias.jpg}
%\end{center}
%Fuente: Google \\
%\hyperlink{uno}{Volver} 
%\end{frame}

%\begin{frame} \label{hcuatro}
%\frametitle{Correlación versus Causalidad}
%\begin{center}
%    \includegraphics[scale=0.35]{Slides Principios de Economia/Figures/Introduccion_1.3_desempleoestres.jpg}
%\end{center}
%Fuente: Google \\
%\hyperlink{uno}{Volver} 
%\end{frame}

%\begin{frame} \label{hcinco}
%\frametitle{Correlación versus Causalidad}
%\begin{center}
%    \includegraphics[scale=0.25]{Slides Principios de Economia/Figures/Introduccion_1.6_errormedicion.jpg}
%\end{center}
%Fuente: Google \\
%\hyperlink{uno}{Volver} 
%\end{frame}

\begin{frame} \label{hdos}
\frametitle{Experimentos}
\begin{center}
\includegraphics[scale=0.1]{Slides Principios de Economia/Figures/rct_vacunas.png}
\end{center}
“Improving immunisation coverage in rural India: clustered randomised controlled evaluation of immunization campaigns with and without incentives” - Banerjee, Duflo, Glennerster \& Kothari (2010) \\  \vspace{2mm}

\end{frame}

\begin{frame} 
\frametitle{Experimentos naturales}
\begin{center}
    \includegraphics[scale=0.12]{Slides Principios de Economia/Figures/Clase1_1.8.jpg}
\end{center}
\end{frame}

%\begin{frame} \label{hocho}
%\frametitle{Experimentos naturales}
%\begin{center}
%    \includegraphics[scale=0.8]{Slides Principios de Economia/Figures/Rossi (1).png}
%\end{center}
%\hyperlink{unob}{Volver} 
%\end{frame}

%\begin{frame} \label{hnueve}
%\frametitle{Diferencias en Diferencias}
%\begin{center}
%    \includegraphics[scale=0.4]{Slides Principios de Economia/Figures/4.9 (1).pdf}
%\end{center}
%\end{frame}

%\begin{frame} 
%\frametitle{Diferencias en Diferencias}
%\begin{center}
%    \includegraphics[scale=0.5]{Slides Principios de Economia/Figures/Experimentowagepng (1).png}
%\end{center}
%\hyperlink{unob}{Volver} 
%\end{frame}

%\begin{frame} \label{hdiez}
%\frametitle{Control sintético}
%\begin{center}
%    \includegraphics[scale=0.7]{Slides Principios de Economia/Figures/abadie (1) (1).png}
%\end{center}
%\hyperlink{unob}{Volver} 
%\end{frame}

\end{document}

