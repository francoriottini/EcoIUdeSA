\documentclass{beamer}
\usepackage{amsmath}
\usepackage[english]{babel} %set language; note: after changing this, you need to delete all auxiliary files to recompile
\usepackage[utf8]{inputenc} %define file encoding; latin1 is the other often used option
\usepackage{csquotes} % provides context sensitive quotation facilities
\usepackage{graphicx} %allows for inserting figures
\usepackage{booktabs} % for table formatting without vertical lines
\usepackage{textcomp} % allow for example using the Euro sign with \texteuro
\usepackage{stackengine}
\usepackage{wasysym}
\usepackage{tikzsymbols}
\usepackage{textcomp}
\usepackage{xcolor}
\usepackage[dvipsnames]{xcolor}
\usepackage{colortbl}
\usepackage{adjustbox}
\usepackage{tikz}
\usetikzlibrary{arrows.meta, calc, decorations.pathreplacing}


\usepackage{amssymb}
\usepackage{multirow}

% Colores
\definecolor{rojo}{RGB}{221, 36, 36}
\definecolor{celeste}{RGB}{173, 216, 230}

\newcommand{\up}{\textcolor{blue}{\Large$\uparrow$}}
\newcommand{\down}{\textcolor{red}{\Large$\downarrow$}}
\newcommand{\question}{\textcolor{red}{\Large\textbf{?}}}


% ELIMINAR COMANDOS DE NAVEGACION%%%%%%%%%%%
\setbeamertemplate{navigation symbols}

%\newcommand{\bubblethis}[2]{
 %       \tikz[remember picture,baseline]{\node[anchor=base,inner sep=0,outer sep=0]%
 %       (#1) {\underline{#1}};\node[overlay,cloud callout,callout relative pointer={(0.2cm,-0.7cm)},%
 %       aspect=2.5,fill=yellow!90] at ($(#1.north)+(-0.5cm,1.6cm)$) {#2};}%
 %   }%
%\tikzset{face/.style={shape=circle,minimum size=4ex,shading=radial,outer sep=0pt,
 %       inner color=white!50!yellow,outer color= yellow!70!orange}}

%% Some commands to make the code easier
\newcommand{\emoticon}[1][]{%
  \node[face,#1] (emoticon) {};
  %% The eyes are fixed.
  \draw[fill=white] (-1ex,0ex) ..controls (-0.5ex,0.2ex)and(0.5ex,0.2ex)..
        (1ex,0.0ex) ..controls ( 1.5ex,1.5ex)and( 0.2ex,1.7ex)..
        (0ex,0.4ex) ..controls (-0.2ex,1.7ex)and(-1.5ex,1.5ex)..
        (-1ex,0ex)--cycle;}
\newcommand{\pupils}{
  %% standard pupils
  \fill[shift={(0.5ex,0.5ex)},rotate=80] 
       (0,0) ellipse (0.3ex and 0.15ex);
  \fill[shift={(-0.5ex,0.5ex)},rotate=100] 
       (0,0) ellipse (0.3ex and 0.15ex);}

\newcommand{\emoticonname}[1]{
  \node[below=1ex of emoticon,font=\footnotesize,
        minimum width=4cm]{#1};}
\usepackage{scalerel}
\usetikzlibrary{positioning}
\usepackage{xcolor,amssymb}
\newcommand\dangersignb[1][2ex]{%
  \scaleto{\stackengine{0.3pt}{\scalebox{1.1}[.9]{%
  \color{red}$\blacktriangle$}}{\tiny\bfseries !}{O}{c}{F}{F}{L}}{#1}%
}
\newcommand\dangersignw[1][2ex]{%
  \scaleto{\stackengine{0.3pt}{\scalebox{1.1}[.9]{%
  \color{red}$\blacktriangle$}}{\color{white}\tiny\bfseries !}{O}{c}{F}{F}{L}}{#1}%
}
\usepackage{fontawesome} % Social Icons
\usepackage{epstopdf} % allow embedding eps-figures
\usepackage{tikz} % allows drawing figures
\usepackage{amsmath,amssymb,amsthm} %advanced math facilities
\usepackage{lmodern} %uses font that support italic and bold at the same time
\usepackage{hyperref}
\usepackage{tikz}
\hypersetup{
    colorlinks=true,
    linkcolor=blue,
    filecolor=magenta,      
    urlcolor=blue,
}
\usepackage{tcolorbox}
%add citation management using BibLaTeX
\usepackage[citestyle=authoryear-comp, %define style for citations
    bibstyle=authoryear-comp, %define style for bibliography
    maxbibnames=10, %maximum number of authors displayed in bibliography
    minbibnames=1, %minimum number of authors displayed in bibliography
    maxcitenames=3, %maximum number of authors displayed in citations before using et al.
    minnames=1, %maximum number of authors displayed in citations before using et al.
    datezeros=false, % do not print dates with leading zeros
    date=long, %use long formats for dates
    isbn=false,% show no ISBNs in bibliography (applies only if not a mandatory field)
    url=false,% show no urls in bibliography (applies only if not a mandatory field)
    doi=false, % show no dois in bibliography (applies only if not a mandatory field)
    eprint=false, %show no eprint-field in bibliography (applies only if not a mandatory field)
    backend=biber %use biber as the backend; backend=bibtex is less powerful, but easier to install
    ]{biblatex}
\addbibresource{../mybibfile.bib} %define bib-file located one folder higher


\usefonttheme[onlymath]{serif} %set math font to serif ones

\definecolor{beamerblue}{rgb}{0.2,0.2,0.7} %define beamerblue color for later use

%%% defines highlight command to set text blue
\newcommand{\highlight}[1]{{\color{blue}{#1}}}


%%%%%%% commands defining backup slides so that frame numbering is correct

\newcommand{\backupbegin}{
   \newcounter{framenumberappendix}
   \setcounter{framenumberappendix}{\value{framenumber}}
}
\newcommand{\backupend}{
   \addtocounter{framenumberappendix}{-\value{framenumber}}
   \addtocounter{framenumber}{\value{framenumberappendix}}
}

%%%% end of defining backup slides

%Specify figure caption, see also http://tex.stackexchange.com/questions/155738/caption-package-not-working-with-beamer
\setbeamertemplate{caption}{\insertcaption} %redefines caption to remove label "Figure".
%\setbeamerfont{caption}{size=\scriptsize,shape=\itshape,series=\bfseries} %sets figure  caption bold and italic and makes it smaller


\usetheme{Boadilla}

%set options of hyperref package
\hypersetup{
    bookmarksnumbered=true, %put section numbers in bookmarks
    naturalnames=true, %use LATEX-computed names for links
    citebordercolor={1 1 1}, %color of border around cites, here: white, i.e. invisible
    linkbordercolor={1 1 1}, %color of border around links, here: white, i.e. invisible
    colorlinks=true, %color links
    anchorcolor=black, %set color of anchors
    linkcolor=beamerblue, %set link color to beamer blue
    citecolor=blue, %set cite color to beamer blue
    pdfpagemode=UseThumbs, %set default mode of PDF display
    breaklinks=true, %break long links
    pdfstartpage=1 %start at first page
    }

\newtcolorbox{boxA}{
    fontupper = \bf,
    boxrule = 1.5pt,
    colframe = black % frame color
}
\newtcolorbox{boxB}{
    boxrule = 1.5pt,
    colframe = blue!70!black,, % frame color
    colback = blue!7!white,
}

% --------------------
% Overall information
% --------------------
\title[Economía I]{Economía I \vspace{3mm}
\\ Magistral 13 \vspace{3mm} \\ Repaso}
\date{}
\author[Victoria Rosino]{Victoria Rosino}
\vspace{0.3cm}
\institute[]{Universidad de San Andrés} 

\begin{document}

\begin{frame}
\vspace{0.3cm}
\titlepage
\centering
\vspace{-0.9cm}
\includegraphics[scale=0.3]{Slides Principios de Economia/Figures/udesa_logo.jpg} 
\end{frame}



\begin{frame}
\frametitle{Repaso: Teoría de Juegos}
    ¿Cuál es la \textbf{mejor respuesta} de cada individuo?
    \begin{itemize}
        \item  Se denomina mejor respuesta a aquella estrategia que proporciona al jugador el pago más elevado, condicional a lo que se conjetura que harán los demás
    \end{itemize}
    \vspace{2mm}
    ¿Qué es una \textbf{estrategia dominante}?
    \begin{itemize}
        \item  Una estrategia dominante es una mejor respuesta a todas las posibles estrategias de otro jugador. Si un agente tiene una estrategia dominante \textbf{siempre} la va a jugar, independientemente de la estrategia que juegue el rival.
        \item Si se encuentra un resultado donde cada individuo juega su estrategia dominante, entonces nos encontramos con un equilibrio en estrategia dominante.
        \end{itemize}
\end{frame}

\begin{frame}
\frametitle{Repaso: Teoría de Juegos}
\textbf{Equilibrio de Nash}
    \begin{itemize}
        \item  Un perfil de estrategias es un equilibrio de Nash si, dadas las estrategias del rival, cada uno de los agentes está jugando su mejor respuesta.
    \end{itemize}
    \begin{boxB}
        \centering
        En un equilibrio de Nash, ninguno de los jugadores tiene incentivos a desviarse, debido a que no puede obtener un pago mayor cambiando de estrategia.
    \end{boxB}
   \textbf{ Equilibrio Pareto eficiente}
        \begin{itemize}
        \item   En un equilibrio Pareto eficiente, no hay otra situación o alternativa en donde uno esté mejor y ninguno esté peor. 
    \end{itemize}

\end{frame}

\begin{frame}{Ejercicio - Teoría de Juegos}
Dos empresas rivales, A y B, venden productos similares. Ambas están evaluando si gastar en una costosa campaña publicitaria o no. La publicidad aumenta significativamente sus ventas si la otra empresa no hace publicidad. Pero si ambas hacen publicidad, el gasto se neutraliza y las ganancias son menores. Si ninguna hace publicidad, ambas mantienen sus clientes actuales.
    \begin{itemize}
        \item El costo de hacer publicidad es de \$3.000 para la empresa A y \$2.000 para la empresa B.
        \item Si \textbf{solo una empresa} hace publicidad, esa empresa gana \$10.000 en ventas, mientras que la otra solo gana \$2.000.
        \item Si \textbf{ambas hacen publicidad}, cada una gana \$6.000 en ventas.
        \item Si \textbf{ninguna hace publicidad}, cada una gana \$5.000.
    \end{itemize}
\end{frame}

\begin{frame}
\frametitle{Ejercicio - Teoría de Juegos}
\begin{enumerate}
    \item Construir la matriz de pagos con las ganancias netas.
    \item Identificar, si lo hay, el Equilibrio de Nash.
    \item ¿Hay estrategias dominantes para alguna de las empresas?
    \item Si hay equilibrio de Nash ¿es Pareto eficiente?
\end{enumerate}
\end{frame}

\begin{frame}
\frametitle{Ejercicio - Teoría de Juegos}
\begin{center}
\renewcommand{\arraystretch}{1.8} % <-- Espaciado vertical
\begin{tabular}{c|c|c}
 & \textbf{B: Publicidad} & \textbf{B: No Publicidad} \\
\hline
\textbf{A: Publicidad} & (3.000,\ 4.000) & (7.000,\ 2.000) \\
\textbf{A: No Publicidad} & (2.000,\ 8.000) & (5.000,\ 5.000) \\
\end{tabular}
\end{center}

\end{frame}

\begin{frame}
\frametitle{Ping Pong de Verdaderos/Falsos}
\centering
Si el objetivo de la empresa fuera maximizar los ingresos totales, la empresa debería disminuir el precio si se encuentra en un punto de la curva de demanda con elasticidad precio mayor a 1
\end{frame}

\begin{frame}
\frametitle{Ping Pong de Verdaderos/Falsos}
\centering
A una empresa en un mercado de competencia perfecta nunca le conviene seguir produciendo si su precio es menor al costo medio total porque tendría pérdidas
\end{frame}

\begin{frame}
\frametitle{Ping Pong de Verdaderos/Falsos}
\centering
El efecto sustitución siempre actúa en dirección opuesta al cambio de precio, mientras que el efecto ingreso puede reforzarlo o contrarrestarlo.
\end{frame}


\begin{frame}
\frametitle{Ping Pong de Verdaderos/Falsos}
\centering
En todo el mundo se comen hamburguesas. Sin embargo, las campañas que promueven el vegetarianismo están afectando su popularidad entre los consumidores. A la vez, el precio de la carne aumentó debido a restricciones productivas. Esto genera que tanto el precio como la cantidad de equilibrio de hamburguesas caigan. 
\end{frame}



\end{document}
